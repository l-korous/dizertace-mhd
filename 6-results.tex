\chapter{Results}
In this section, results from the computation using the implemented software are presented. There are two classical benchmarks for 3-dimensional MHD equations, namely the MHD Blast \cite{blast1}, \cite{blast2}, and the Orszag-Tang vortex \cite{vortex}. And then the main section of this work contains results from the flux rope eruption model on and above the Sun's surface.

\section{Benchmarks}
The benchmarks presented in this Section do not have an exact analytical solution, but the formation of waves and discontinuities is well studied, and benchmarking is usually performed on the basis of comparing the structure and presence of non-physical attributes.
\subsection{MHD Blast}
\subsubsection{MHD Blast - original version}
This benchmark has been used for decades - \cite{blast0}, \cite{blast1}, \cite{blast2} - in a variety of configurations. The setup as described in \cite{blast0}, \cite{blast1} is defined, although in \cite{blast1} with interchanged $x-$ and $y-$ coordinates, by the initial conditions:
\begin{eqnarray}
\label{mhdBlastOld}
\gamma & = & 5 / 3\\ \nonumber
p_0\lo\bfx, t\ro & = & 10\ \ \text{for}\ \left|\bfx\right| < 0.1\\ \nonumber
p_0\lo\bfx, t\ro & = & 0.1\ \ \text{for}\ \left|\bfx\right| \geq 0.1\\ \nonumber
\rho\lo\bfx, t = 0\ro & = & 1,\\ \nonumber
p\lo\bfx, t = 0\ro & = & p_0\lo\bfx, t\ro,\\ \nonumber
\bfu_1\lo\bfx, t = 0\ro & = & 0,\\ \nonumber
\bfu_2\lo\bfx, t = 0\ro & = & 0,\\ \nonumber
\bfu_3\lo\bfx, t = 0\ro & = & 0,\\ \nonumber
\bfB_1\lo\bfx, t = 0\ro & = & 0,\\ \nonumber
\bfB_2\lo\bfx, t = 0\ro & = & 100,\\ \nonumber
\bfB_3\lo\bfx, t = 0\ro & = & 0.
\end{eqnarray}
Total energy is calculated using \ref{magU}, \ref{kinU}, \ref{presU}.
It is obvious from the initial setup, that the example is true to its name, and it is in fact a blast of the overpressured area $\left|\bfx\right| < 0.1$, where pressure $p$ is 100$\times$ larger than elsewhere in the domain. This setup is completed with simple outflow boundary condition.


\subsubsection{MHD Blast - extended version}
\label{sec:blast}
An extended version of the benchmark, used in \cite{blastNew1}, \cite{athenaBlast}, and a similar problem was used also in \cite{blastNew2}. The initial conditions are a little different than in the case of \ref{mhdBlastOld}, and read:
\begin{eqnarray}
\label{mhdBlastNew}
\gamma & = & 5 / 3\\ \nonumber
p_0\lo\bfx, t\ro & = & 10\ \ \text{for}\ \left|\bfx\right| < 0.1\\ \nonumber
p_0\lo\bfx, t\ro & = & 0.1\ \ \text{for}\ \left|\bfx\right| \geq 0.1\\ \nonumber
\rho\lo\bfx, t = 0\ro & = & 1,\\ \nonumber
p\lo\bfx, t = 0\ro & = & p_0\lo\bfx, t\ro,\\ \nonumber
\bfu_1\lo\bfx, t = 0\ro & = & 0,\\ \nonumber
\bfu_2\lo\bfx, t = 0\ro & = & 0,\\ \nonumber
\bfu_3\lo\bfx, t = 0\ro & = & 0,\\ \nonumber
\bfB_1\lo\bfx, t = 0\ro & = & \frac{1}{\sqrt{2}},\\ \nonumber
\bfB_2\lo\bfx, t = 0\ro & = & \frac{1}{\sqrt{2}},\\ \nonumber
\bfB_3\lo\bfx, t = 0\ro & = & 0.
\end{eqnarray}



\subsection{Orszag-Tang vortex}
This problem was first described in \cite{vortex} and has been extensively used as a benchmark for 2- and 3- dimensional MHD code. It is a simple model of the evolution of MHD turbulence including interactions between the several shock waves that appear. The Orszag-Tang system is defined by the initial conditions:
\begin{eqnarray}
\rho_0 & = & \frac{25}{36 \pi}\\ \nonumber
p_0 & = & \frac{5}{12 \pi}\\  \nonumber
B_0 & = & \sqrt{\frac{1}{4 \pi}}\\ \nonumber
\gamma & = & 5 / 3\\ \nonumber
\rho\lo\bfx, t = 0\ro & = & \rho_0,\\ \nonumber
p\lo\bfx, t = 0\ro & = & p_0,\\ \nonumber
\bfu_1\lo\bfx, t = 0\ro & = & -\sin(2 \pi y),\\ \nonumber
\bfu_2\lo\bfx, t = 0\ro & = & \sin(2 \pi x),\\ \nonumber
\bfu_3\lo\bfx, t = 0\ro & = & 0,\\ \nonumber
\bfB_1\lo\bfx, t = 0\ro & = & -B_0 \sin(2 \pi y),\\ \nonumber
\bfB_2\lo\bfx, t = 0\ro & = & B_0 \sin(4 \pi x),\\ \nonumber
\bfB_3\lo\bfx, t = 0\ro & = & 0,\\ \nonumber
U\lo\bfx, t = 0\ro & = & \frac{p_0}{\gamma - 1} + U_m\lo\bfx, t\ro + U_k\lo\bfx, t\ro,
\end{eqnarray}
where the last term is an application of \ref{magU}, \ref{kinU}, \ref{presU}.

This configuration is strongly unstable, leading to a wide spectrum of propagating MHD modes and shock waves.

TODO - ze clanku papers- Londrillo.pdf vzit stranu 30 a napocitat to stejny

\section{Flux rope eruption model}
This model is based on the original Titov-Demoulin model from \cite{td}, as used in \cite{miraClanek}.

The model parameters are as follows. Note that $k_B$ is the Boltzmann constant $k_B = 1.38064852 \times 10^{-23} \frac{\mathrm{J}}{\mathrm{K}}$, $m_p$ is the plasma mass, and $g$ gravitational acceleration.

\begin{eqnarray}
\beta & = & 0.05\ \ \ \ ...\ \text{Plasma beta}\\
L_G & = & 2\ k_B \frac{T_{ext}}{\lo m_p g \ro} = 1.2 \times 10^8 \left[\text{m}\right]\\
L_G & = & 20 \ \ \ \ ...\ \text{Coronal height scale in dimension-less units}\\
\\
& !!! & \text{TODO - Proc je v kodu} L_G = 0.0 \ ?\
\\ \\
N_t & = &-3\ \ \ \ ...\ \text{Torus winding number}\\
R & = &4\ \ \ \ ...\ \text{Torus major radius}\\
2L & = & 4\ \ \ \ ...\ \text{Magnetic charge separation distance}\\
d & = & 2\ \ \ \ ...\ \text{Geometrical factor}\\
q_{mag} & = & \left| \frac{\ln\lo 8 e^{-5/4} R\ro}{4} N_t \lo\frac{L}{R}\ro^2\left[1 + \lo\frac{R}{L}\ro^2\right]^{3/2}\right|\\
q_{mag} & \approx & \text{Normalised magnetic charge corresponding to global equilibrium}\\
q_{mag} & = & \frac{ \ln\lo 8R \ro - \frac{5}{4}}{4} \left| N_t \right| \frac{\lo 1 + \lo\frac{L}{R}\ro^2\ro \sqrt{1 + \lo\frac{L}{R}\ro^2}}{\frac{L}{R}}\\
\\
& !!! & \text{TODO - Neni toto spatne nakodene?}
\\\ \\
H & = & 2\ \frac{N_t^2}{R^2}\ \ \ \ ...\ \text{"Helicity" factor inside tho loop}\\
\frac{ T_{ext} }{T_{in}} & = & 10\ \ \ \ ...\ \text{Coronal/prominence temperature ratio}\\
\\
& !!! & \text{TODO - Proc je v kodu Tc2Tp = 1?}
\\\ \\
\end{eqnarray}