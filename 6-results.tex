\chapter{Results}
In this section, results from the computation using the implemented software are presented. There are two classical benchmarks for 3-dimensional MHD equations, namely the MHD Blast \cite{blast1}, \cite{blast2}, and the Orszag-Tang vortex \cite{vortex}. And then the main section of this work contains results from the flux tube eruption model on and above the Sun's surface.
\paragraph{}
In most cases, we are interested in the distribution of plasma density $\rho$ and the magnetic field $B$ in the domain.

\section{Benchmarks}

The benchmarks presented in this Section do not have an exact analytical solution, but the formation of waves and discontinuities is well studied, and benchmarking is usually performed on the basis of comparing the structure and presence of non-physical attributes.
\subsection{Hardware specification}
For all following benchmarks, the setup described below was used. The computational mesh $T_h$ was formed in all cases by rectangular hexahedra.
The value of the time step $\tau$ used was set according to the CFL condition (\Cref{section:CFL}).
The Taylor basis functions (see \cite{KuzminVertex}) and Divergence-free basis functions (\Cref{tbl:divFreeBasis}) of order $0$ (piecewise constant functions) and $1$ (piecewise linear functions) were used.
Where appropriate (for piecewise linear basis functions), the slope limiting technique from \Cref{sec:vertex} was used.
Illustration of the obtained results follows below - these results were obtained using one node of the department's computational cluster with these parameters:
\begin{itemize}
    \item CPU: Intel(R) Xeon(R) CPU E5-2680 v3 @ 2.50GHz,
    \item \# of Cores: 48 per node (nodes 1, 2), 16 per node (nodes 3, 4),
    \item Vectorization support: AVX2,
    \item Parallelization implemented: Intel TBB,
    \item Distributed calculation implemented: OpenMPI,
    \item RAM: 512 GB,
    \item C/C++ Compiler: GNU gcc 5.4.0 .
\end{itemize}

\subsection{MHD Blast}
\subsubsection{MHD Blast - original version}
This benchmark has been used for decades - \cite{blast0}, \cite{blast1}, \cite{blast2} - in a variety of configurations and as a benchmark in software - e.g. \citep{athena}. The setup as described in \cite{blast0}, \cite{blast1} is defined (although in \cite{blast1} with interchanged $x-$ and $y-$ coordinates) by the initial conditions:
\begin{align}
\label{mhdBlastOld}
\gamma & =  5 / 3\\ \nonumber
p_0\lo\bfx, t\ro & =  100\ \ \text{for}\ \left|\bfx\right| < 0.1\\ \nonumber
p_0\lo\bfx, t\ro & =  1\ \ \text{for}\ \left|\bfx\right| \geq 0.1\\ \nonumber
\rho\lo\bfx, t = 0\ro & =  1,\\ \nonumber
p\lo\bfx, t = 0\ro & =  p_0\lo\bfx, t\ro,\\ \nonumber
\bfu_1\lo\bfx, t = 0\ro & =  0,\\ \nonumber
\bfu_2\lo\bfx, t = 0\ro & =  0,\\ \nonumber
\bfu_3\lo\bfx, t = 0\ro & =  0,\\ \nonumber
\bfB_1\lo\bfx, t = 0\ro & =  0,\\ \nonumber
\bfB_2\lo\bfx, t = 0\ro & =  100,\\ \nonumber
\bfB_3\lo\bfx, t = 0\ro & =  0.
\end{align}
Total energy is calculated using \Cref{magU}, \Cref{kinU}, \Cref{presU}.
The domain $\Omega$ is a square, with scaling quite arbitrarily used in the papers. In the case of \cite{blast1}, $\Omega = [0, 1] \times [0, 1]$, in this work it is $\Omega = [-0.25, 0.25] \times [-0.25, 0.25]$.

It is obvious from the initial setup, that the example is true to its name, and it is in fact a blast of the over-pressured area $\left|\bfx\right| < 0.1$, where pressure $p$ is 100$\times$ larger than elsewhere in the domain. This setup is completed with simple outflow boundary condition \Cref{bcoutdef}.
In the next figures, the solution as presented in \cite{blast1} is compared to the solution obtained with the approach described in this work. Note that the solution from \cite{blast1} needed to have the axes transformed $\lo x\rightleftharpoons y\ro$ with respect to the original paper. The figure \Cref{figure:blastOldRef} is taken from the article \cite{blast1} from the page 33. Unfortunately the paper does not specify the precise time at which the snapshots are taken.

\begin{figure}[H]
\centering
\hspace{-8mm}
\subfigure{\includegraphics[width=0.4\textwidth]{img/mhd-blast/old/ref.jpg}}
\subfigure{\includegraphics[width=0.395\textwidth]{img/mhd-blast/old/refmag.jpg}}
\caption{Results from \cite{blast1}, density(left), magnetic energy(right)}
\label{figure:blastOldRef}
\end{figure}

\vspace{-5mm}
\begin{figure}[H]
	\begin{center}
		\includegraphics[width=0.87\textwidth]{img//mhd-blast/old/mynew1.jpg}
	\caption{Obtained results, $t = 1\times 10^{-3}$, density(left), magnetic energy(right)}
	\label{figure:blastOldMy1}
	\end{center}
\end{figure}
\vspace{-8mm}

\begin{figure}[H]
	\begin{center}
		\includegraphics[width=0.87\textwidth]{img//mhd-blast/old/mynew2.jpg}
	\caption{Obtained results, $t = 6\times 10^{-3}$, density(left), magnetic energy(right)}
	\label{figure:blastOldMy2}
	\end{center}
\end{figure}
\vspace{-8mm}

\begin{figure}[H]
	\begin{center}
		\includegraphics[width=0.87\textwidth]{img//mhd-blast/old/mynew3.jpg}
	\caption{Obtained results, $t = 11\times 10^{-3}$, density(left), magnetic energy(right)}
	\label{figure:blastOldMy3}
	\end{center}
\end{figure}
\vspace{-8mm}

\begin{figure}[H]
	\begin{center}
		\includegraphics[width=0.87\textwidth]{img//mhd-blast/old/mynew4.jpg}
	\caption{Obtained results, $t = 16\times 10^{-3}$, density(left), magnetic energy(right)}
	\label{figure:blastOldMy4}
	\end{center}
\end{figure}
\vspace{-8mm}

\begin{figure}[H]
	\begin{center}
		\includegraphics[width=0.87\textwidth]{img//mhd-blast/old/mynew5.jpg}
	\caption{Obtained results, $t = 23\times 10^{-3}$, density(left), magnetic energy(right)}
	\label{figure:blastOldMy5}
	\end{center}
\end{figure}
\vspace{-5mm}
The solution in \Cref{figure:blastOldMy4}, \Cref{figure:blastOldMy5}  is apparently almost identical to that in \Cref{figure:blastOldRef}. To demonstrate the distributed nature of the computation, in \Cref{figure:subdomainsBlastOld}, color-mapping of elements $K \in T$ to processors owning the particular element is presented (see \Cref{section:ditributedTria} for details). Note that there were $48$ processors used for the computation.

\begin{figure}[H]
	\begin{center}
		\includegraphics[width=0.5\textwidth]{img//mhd-blast/old/subdomain.jpg}
	\caption{Color-mapping of elements to processors for the computation of \Crefrange{figure:blastOldMy1}{figure:blastOldMy5}}
	\label{figure:subdomainsBlastOld}
	\end{center}
\end{figure}
\vspace{-5mm}

This however, was a static calculation with 200 mesh elements in the $x-$ and $y-$ dimensions. In order to see how the AMR (see \Cref{chapter:amr}) performs, the same computation was performed in adaptive setup. Starting from a very coarse mesh of 10 elements in both dimensions, the evolving mesh and its distribution to processors are shown in \Crefrange{figure:blastOldMyAdapt1}{figure:blastOldMyAdapt6} - the distribution to processors on the left, the mesh elements on the right.

\begin{figure}[H]
	\begin{center}
		\includegraphics[width=0.87\textwidth]{img//mhd-blast/old/mya1.jpg}
	\caption{Obtained results, initial state, distribution of $\rho$ with highlighted elements (right) and distribution of elements to processors (left)}
	\label{figure:blastOldMyAdapt1}
	\end{center}
\end{figure}
\vspace{-8mm}

\begin{figure}[H]
	\begin{center}
		\includegraphics[width=0.87\textwidth]{img//mhd-blast/old/mya2.jpg}
	\caption{Obtained results, $t = 5\times 10^{-3}$, distribution of $\rho$ with highlighted elements (right) and distribution of elements to processors (left)}
	\label{figure:blastOldMyAdapt2}
	\end{center}
\end{figure}
\vspace{-8mm}

\begin{figure}[H]
	\begin{center}
		\includegraphics[width=0.87\textwidth]{img//mhd-blast/old/mya3.jpg}
	\caption{Obtained results, $t = 10\times 10^{-3}$, distribution of $\rho$ with highlighted elements (right) and distribution of elements to processors (left)}
	\label{figure:blastOldMyAdapt3}
	\end{center}
\end{figure}
\vspace{-8mm}

\begin{figure}[H]
	\begin{center}
		\includegraphics[width=0.87\textwidth]{img//mhd-blast/old/mya4.jpg}
	\caption{Obtained results, $t = 15\times 10^{-3}$, distribution of $\rho$ with highlighted elements (right) and distribution of elements to processors (left)}
	\label{figure:blastOldMyAdapt4}
	\end{center}
\end{figure}
\vspace{-8mm}

\begin{figure}[H]
	\begin{center}
		\includegraphics[width=0.87\textwidth]{img//mhd-blast/old/mya5.jpg}
	\caption{Obtained results, $t = 20\times 10^{-3}$, distribution of $\rho$ with highlighted elements (right) and distribution of elements to processors (left)}
	\label{figure:blastOldMyAdapt5}
	\end{center}
\end{figure}
\vspace{-8mm}

\begin{figure}[H]
	\begin{center}
		\includegraphics[width=0.87\textwidth]{img//mhd-blast/old/mya6.jpg}
	\caption{Obtained results, $t = 25\times 10^{-3}$, distribution of $\rho$ with highlighted elements (right) and distribution of elements to processors (left)}
	\label{figure:blastOldMyAdapt6}
	\end{center}
\end{figure}
\vspace{-8mm}

\subsubsection{MHD Blast - extended version}
\label{sec:blastNew}
An extended version of the benchmark has been used in \cite{blastNew1}, \cite{athenaBlast}, and a similar problem was used also in \cite{blastNew2} - the description of the benchmark is also available at:\url{http://www.astro.princeton.edu/~jstone/Athena/tests/blast/blast.html}.
In this version, the domain dimensions are set as a rectangle: $\Omega = [-0.5, 0.5] \times [-0.75, 0.75]$.
The initial conditions are a little different than in the case of \Cref{mhdBlastOld}, and read:
\begin{align}
\label{mhdBlastNew}
\gamma & =  5 / 3\\ \nonumber
p_0\lo\bfx, t\ro & =  10\ \ \text{for}\ \left|\bfx\right| < 0.1\\ \nonumber
p_0\lo\bfx, t\ro & =  0.1\ \ \text{for}\ \left|\bfx\right| \geq 0.1\\ \nonumber
\rho\lo\bfx, t = 0\ro & =  1,\\ \nonumber
p\lo\bfx, t = 0\ro & =  p_0\lo\bfx, t\ro,\\ \nonumber
\bfu_1\lo\bfx, t = 0\ro & =  0,\\ \nonumber
\bfu_2\lo\bfx, t = 0\ro & =  0,\\ \nonumber
\bfu_3\lo\bfx, t = 0\ro & =  0,\\ \nonumber
\bfB_1\lo\bfx, t = 0\ro & =  \frac{1}{\sqrt{2}},\\ \nonumber
\bfB_2\lo\bfx, t = 0\ro & =  \frac{1}{\sqrt{2}},\\ \nonumber
\bfB_3\lo\bfx, t = 0\ro & =  0,
\end{align}
with periodic boundary conditions on the top-bottom, and left-right parts of the boundary. That is, with respect to \Cref{periodicMapping}, the two pairs $\Gamma_1, \Gamma_2, \Gamma_1^{'}, \Gamma_2^{'}$ are specified as follows:
\begin{align}
\Gamma_1 & = \left\{-0.5\right\} \times [-0.75, 0.75],\\
\Gamma_2 & = \left\{0.5\right\} \times [-0.75, 0.75],
\end{align}
and
\begin{align}
\Gamma_1^{'} & = [-0.5, 0.5] \times \left\{-0.75\right\},\\
\Gamma_2^{'} & = [-0.5, 0.5] \times \left\{0.75\right\}.
\end{align}

\subsubsection{Results}
The first set of results are from calculations using piecewise-constant elements, on three successively uniformly refined meshes. The meshes used for computations \cref{figure:blastNew01}{figure:blastNew05} contained:
\begin{itemize}
\item $100 \times 150$ elements (left)
\item $200 \times 300$ elements (middle)
\item $400 \times 600$ elements (right)
\end{itemize}

\begin{figure}[H]
	\begin{center}
		\includegraphics[width=0.95\textwidth]{img//mhd-blast/new/blast,noadapt1.jpg}
	\caption{Obtained results, $t \approx 0.1$, distribution of $\rho$ (top), with line distribution along bottom-left $\rightarrow$ top-right diagonal (bottom)}
	\label{figure:blastNew01}
	\end{center}
\end{figure}
\vspace{-8mm}

\begin{figure}[H]
	\begin{center}
		\includegraphics[width=0.95\textwidth]{img//mhd-blast/new/blast,noadapt3.jpg}
	\caption{Obtained results, $t = \approx 0.2$, distribution of $\rho$ (top), with line distribution along bottom-left $\rightarrow$ top-right diagonal (bottom)}
	\label{figure:blastNew02}
	\end{center}
\end{figure}
\vspace{-8mm}

\begin{figure}[H]
	\begin{center}
		\includegraphics[width=0.95\textwidth]{img//mhd-blast/new/blast,noadapt5.jpg}
	\caption{Obtained results, $t \approx 0.3$, distribution of $\rho$ (top), with line distribution along bottom-left $\rightarrow$ top-right diagonal (bottom)}
	\label{figure:blastNew03}
	\end{center}
\end{figure}
\vspace{-8mm}

\begin{figure}[H]
	\begin{center}
		\includegraphics[width=0.95\textwidth]{img//mhd-blast/new/blast,noadapt14.jpg}
	\caption{Obtained results, $t = \approx 0.75$, distribution of $\rho$ (top), with line distribution along bottom-left $\rightarrow$ top-right diagonal (bottom)}
	\label{figure:blastNew03}
	\end{center}
\end{figure}
\vspace{-8mm}

\begin{figure}[H]
	\begin{center}
		\includegraphics[width=0.95\textwidth]{img//mhd-blast/new/blast,noadapt18.jpg}
	\caption{Obtained results, $t \approx 0.95$, distribution of $\rho$ (top), with line distribution along bottom-left $\rightarrow$ top-right diagonal (bottom)}
	\label{figure:blastNew03}
	\end{center}
\end{figure}
\vspace{-8mm}

It is clearly visible, that the solution is somehow smeared, and the uniform mesh refinements improve the situation, but not greatly, and for a large cost of storage size for storing much more mesh elements.
\paragraph{}
In order to amend the situation, and be able to obtain a higher-quality solution with a reasonable number of mesh elements, piecewise-linear basis functions need to be used. Results with piecewise-linear basis functions, on three successively uniformly refined meshes are given in \cref{figure:blastNew11}{figure:blastNew15}. The meshes for these computations contained:
\begin{itemize}
\item $100 \times 150$ elements (left)
\item $200 \times 300$ elements (middle)
\item $400 \times 600$ elements (right)
\end{itemize}


TODO - moje vysledky Blastu s adaptivitou\\

\subsection{Orszag-Tang vortex}
This problem was first described in \cite{vortex} and has been extensively used as a benchmark for 2- and 3- dimensional MHD code (\cite{blast0}, \cite{blast1}, \cite{honzaFem}, \cite{otnew}, and many others). It is a simple model of the evolution of MHD turbulence including interactions between the several shock waves that appear. The Orszag-Tang system is defined by the initial conditions:
\begin{align}
\rho_0 & =  \frac{25}{36 \pi}\\ \nonumber
p_0 & =  \frac{5}{12 \pi}\\  \nonumber
B_0 & =  \sqrt{\frac{1}{4 \pi}}\\ \nonumber
\gamma & =  5 / 3\\ \nonumber
\rho\lo\bfx, t = 0\ro & =  \rho_0,\\ \nonumber
p\lo\bfx, t = 0\ro & =  p_0,\\ \nonumber
\bfu_1\lo\bfx, t = 0\ro & =  -\sin(2 \pi y),\\ \nonumber
\bfu_2\lo\bfx, t = 0\ro & =  \sin(2 \pi x),\\ \nonumber
\bfu_3\lo\bfx, t = 0\ro & =  0,\\ \nonumber
\bfB_1\lo\bfx, t = 0\ro & =  -B_0 \sin(2 \pi y),\\ \nonumber
\bfB_2\lo\bfx, t = 0\ro & =  B_0 \sin(4 \pi x),\\ \nonumber
\bfB_3\lo\bfx, t = 0\ro & =  0,\\ \nonumber
U\lo\bfx, t = 0\ro & =  \frac{p_0}{\gamma - 1} + U_m\lo\bfx, t\ro + U_k\lo\bfx, t\ro,
\end{align}
where the last term is an application of \Cref{magU}, \Cref{kinU}, \Cref{presU}. The domain $\Omega$ is set as $\Omega = [0, 1] \times [0, 1]$ and the problem is equipped with periodic boundary conditions on the top-bottom, and left-right parts of the boundary, similarly as in \Cref{sec:blastNew}. This configuration is strongly unstable, leading to a wide spectrum of propagating MHD modes and shock waves.
\paragraph{}
As before, in order to compare with reference papers, figures are presented from these papers - see \Cref{figure:otRef}. The images are taken from pages 30 (\cite{blast1}), and 20/282 (\cite{blast0}) respectively. All are taken at $t = 0.5$.

\begin{figure}[H]
\centering
\subfigure{\vspace{12mm}\includegraphics[width=0.4\textwidth]{img/ot/ref-londrillo-pressure.jpg}}
\subfigure{\includegraphics[width=0.4\textwidth]{img/ot/ref-zachary-pressure-contour.jpg}}\\
\subfigure{\includegraphics[width=0.75\textwidth]{img/ot/ref-zachary-pressure-profile.jpg}}
\caption{$p$ isolines from \cite{blast1} (top left), $p$ isolines from \cite{blast0} (top right), $p$ along $y = 0.3125$ from \cite{blast0} (bottom).}
\label{figure:otRef}
\end{figure}

Results obtained in the implemented software are presented in \Cref{figure:myOt1}{figure:myOt3}. The result on \Cref{figure:myOt3} on the right is obviously in almost exact accordance with \Cref{figure:otRef}.

\begin{figure}[H]
\centering
\subfigure{\includegraphics[width=0.45\textwidth]{img/ot/my1.jpg}}\hspace{12mm}
\subfigure{\includegraphics[width=0.45\textwidth]{img/ot/my2.jpg}}
\caption{$p$ distribution and $p$ along $y = 0.3125$, $t = 0.03$ (left), $t = 0.18$ (right)}
\label{figure:myOt1}
\end{figure}
\vspace{-8mm}
\begin{figure}[H]
\centering
\subfigure{\includegraphics[width=0.45\textwidth]{img/ot/my3.jpg}}\hspace{12mm}
\subfigure{\includegraphics[width=0.45\textwidth]{img/ot/my4.jpg}}
\caption{$p$ distribution and $p$ along $y = 0.3125$, $t = 0.28$ (left), $t = 0.38$ (right)}
\label{figure:myOt2}
\end{figure}
\vspace{-8mm}
\begin{figure}[H]
\centering
\subfigure{\includegraphics[width=0.45\textwidth]{img/ot/my5.jpg}}\hspace{12mm}
\subfigure{\includegraphics[width=0.45\textwidth]{img/ot/my6.jpg}}
\caption{$p$ distribution and $p$ along $y = 0.3125$, $t = 0.48$ (left), $t = 0.5$ (right)}
\label{figure:myOt3}
\end{figure}
\vspace{-8mm}

\ \\
TODO - moje vysledky O-T s adaptivitou\\

\section{Flux tube eruption model}
This model is based on the original Titov-Demoulin model from \cite{td}, as used in \cite{miraClanek}.

The model parameters are as follows. Note that $k_B$ is the Boltzmann constant $k_B = 1.38064852 \times 10^{-23} \frac{\mathrm{J}}{\mathrm{K}}$, $m_p$ is the plasma mass, and $g$ gravitational acceleration. Parameter values read
\begin{align}
\nonumber \beta & =  0.05\ \ \ \ ...\ \text{Plasma beta},\\
\nonumber L_G & =  2\ k_B \frac{T_{ext}}{\lo m_p g \ro} = 1.2 \times 10^8 \left[\text{m}\right],\\
\nonumber L_G & =  20 \ \ \ \ ...\ \text{Coronal height scale in dimension-less units},\\
\nonumber N_t & = 5\ \ \ \ ...\ \text{Torus winding number},\\
\nonumber R & = 4\ \ \ \ ...\ \text{Torus major radius},\\
\nonumber L & = 1.5\ \ \ \ ...\ \text{Magnetic charge separation distance},\\
\nonumber d & =  1.5\ \ \ \ ...\ \text{Geometrical factor},\\
\nonumber q & =  \frac{\ln\lo 8 e^{-5/4} R\ro}{4} N_t \lo\frac{L}{R}\ro^2\left[1 + \lo\frac{R}{L}\ro^2\right]^{3/2},\\
\nonumber q & \approx \text{Normalised magnetic charge corresponding to global equilibrium},\\
\nonumber H & = 2\ \frac{N_t^2}{R^2}\ \ \ \ ...\ \text{"Helicity" factor inside tho loop},\\
\nonumber \frac{ T_{ext} }{T_{in}} & =  10\ \ \ \ ...\ \text{Coronal/prominence temperature ratio}.\\
\end{align}
The domain $\Omega$ is taken as $\left[-2.5, 2.5\right] \times \left[-5, 5\right] \times \left[0, 5\right]$, and the model is equipped with boundary conditions:
\begin{align}
... & = \Gamma_{ij} \text{on the left, right, front, back, and top boundary},\ z > 0,\\
... & = \Gamma_{ij} \text{on the bottom boundary},\ z = 0.
\end{align}
TODO add proper description of the example
TODO add comparison with the article