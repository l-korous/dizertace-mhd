\chapter{Results}
In this section, results from the computation using the implemented software are presented. There are two classical benchmarks for 3-dimensional MHD equations, namely the MHD Blast \cite{blast1}, \cite{blast2}, and the Orszag-Tang vortex \cite{vortex}. And then the main section of this work contains results from the flux rope eruption model on and above the Sun's surface.

\section{Benchmarks}
The benchmarks presented in this Section do not have an exact analytical solution, but the formation of waves and discontinuities is well studied, and benchmarking is usually performed on the basis of comparing the structure and presence of non-physical attributes.
\subsection{MHD Blast}
\label{sec:blast}


\subsection{Orszag-Tang vortex}

TODO - ze clanku papers- Londrillo.pdf vzit stranu 30 a napocitat to stejny

\section{Flux rope eruption model}
This model is based on the original Titov-Demoulin model from \cite{td}, as used in \cite{miraClanek}.

The model parameters are as follows. Note that $k_B$ is the Boltzmann constant $k_B = 1.38064852 \times 10^{-23} \frac{\mathrm{J}}{\mathrm{K}}$, $m_p$ is the plasma mass, and $g$ gravitational acceleration.

\begin{eqnarray}
\beta & = & 0.05\ \ \ \ ...\ \text{Plasma beta}\\
L_G & = & 2.0\ k_B \frac{T_{ext}}{\lo m_p g \ro} = 1.2 \times 10^8 \left[\text{m}\right]\\
L_G & = & 20.0 \ \ \ \ ...\ \text{Coronal height scale in dimension-less units}\\
\\
& !!! & \text{TODO - Proc je v kodu} L_G = 0.0 \ ?\
\\ \\
N_t & = &-3.0\ \ \ \ ...\ \text{Torus winding number}\\
R & = &4.0\ \ \ \ ...\ \text{Torus major radius}\\
2L & = & 4.0\ \ \ \ ...\ \text{Magnetic charge separation distance}\\
d & = & 2.0\ \ \ \ ...\ \text{Geometrical factor}\\
q_{mag} & = & \left| \frac{\ln\lo 8.0 e^{-5/4} R\ro}{4} N_t \lo\frac{L}{R}\ro^2\left[1 + \lo\frac{R}{L}\ro^2\right]^{3/2}\right|\\
q_{mag} & \approx & \text{Normalised magnetic charge corresponding to global equilibrium}\\
q_{mag} & = & \frac{ \ln\lo 8R \ro - \frac{5}{4}}{4} \left| N_t \right| \frac{\lo 1 + \lo\frac{L}{R}\ro^2\ro \sqrt{1 + \lo\frac{L}{R}\ro^2}}{\frac{L}{R}}\\
\\
& !!! & \text{TODO - Neni toto spatne nakodene?}
\\\ \\
H & = & 2.0\ \frac{N_t^2}{R^2}\ \ \ \ ...\ \text{"Helicity" factor inside tho loop}\\
\frac{ T_{ext} }{T_{in}} & = & 10.0\ \ \ \ ...\ \text{Coronal/prominence temperature ratio}\\
\\
& !!! & \text{TODO - Proc je v kodu Tc2Tp = 1.0?}
\\\ \\
\end{eqnarray}