\section{State of the art}
Only recently, the scientific computation community, due to the advances in computer and supercomputer capabilities, has started with non-trivial numerical simulations of such complex physical phenomena that the MHD model describes. Since both for industrial applications, and obviously for astrophysical application of the MHD model, it is quite expensive (or downright impossible) to perform any experiments, the benefit of being able to simulate the phenomena on a computer is very large.\\
There exist several available numerical simulation codes, such as \cite{athena}, \cite{zeus}, \cite{ramses}, \cite{honzaFD}. These codes have been successfully applied to a range of problems in astrophysics.\\
There are many numerical methods implemented in these codes, such as the finite difference method (\cite{honzaFD}), finite volume method (\cite{ramses}), and the (continous) finite element method (\cite{honzaFem}).

There have been some attempts to employ also the discontinuous Galerkin method (\cite{mhdDg}, \cite{mhdDg2}), but so far no open-source generic software employing this method is available.

The reason for the development of a new code is two-fold. First, there is a unique collaboration between the Astronomical Institute of the Czech Academy of Sciences and the University of West Bohemia, where astrophysicists work together with electrical engineers (from theoretical and numerical modeling backgrounds), and the developed code will be usable for both simulating of astrophysical MHD phenomena, and industrial MHD applications.
\paragraph{}
Second, the newly developed code is based on locally-adaptive Discontinuous Galerkin method, which yields several advantages over the existing codes (which use e.g. finite difference, or finite volumes methods) developed at institutions of such high quality as \emph{Princeton} - \cite{athena}, \cite{zeus}. The advantages are especially of performance, and automation nature - method of higher order together with local mesh refinement yields results qualitatively and quantitatively comparable to low order uniform mesh methods, but with computational cost that can easily be an order of magnitude smaller. Automation is mentioned here related to the AMR, automatic mesh refinement, which, without user interaction, can optimize the computational triangulation for a particular time instance in the evolution of the modeled phenomena.
\paragraph{}
Another benefit (namely over \cite{zeus})of the newly created software are the use of modern object-oriented programming techniques and experience gained on creating finite element software (\cite{ja1}, \cite{ja2}, \cite{ja3}).
The implementation related to this work is written in the C++ language, with the use of existing software packages that are proven, and used by a wide community of researchers all over the world - deal.II (\cite{deal}), UMFPACK (\cite{umfpack}), Paraview(\url{http://www.paraview.org/}).

\paragraph{}
The state of the art of numerical simulation of magnetohydrodynamics can be summarized as a state when the mathematical theory of the equations is quite solid, but the methods to solve the equations numerically in the most optimal and fast way are still being improved. The numerical solution is not merely about theoretical convergence rates and attributes of the particular method, but also the actual implementation plays an important role - i.e. programming, hardware, and software, and execution, both before, during, and very importantly after the actual method invocation (of the so-called \textit{postprocessing} of results). In all aspects of implementation, there is space for new approaches, new ideas, new milestones, that can expand the capabilities of today's numerical solution of magnetohydrodynamics phenomena.