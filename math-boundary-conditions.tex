\section{Boundary conditions}

\subsection{Essential boundary conditions}
Since the solution of the problem specified in (TODO: sekce kde je problem) is discontinous (also on the boundary), we are not able to employ any essential boundary conditions of the form $u\left(x, y, z\right) = u_D\left(x, y, z\right)$ with a known $u_D$.
If such a condition is required from the physical nature of the described phenomenon, it is only implied by the use of numerical flux (TODO: see sekci kde to je) with prescribed values "outside of the simulation domain".


\subsection{Outflow (do-nothing) boundary conditions}
If we want to model free boundary, that is not in any way present as an actual physical boundary or interface, we cannot simply omit specifying anything on the particular boundary (applying zero Neumann condition), since for any volume attached to such boundary, there would be imbalance in e.g. mass, that would flow in / out of the volume through any volume boundary except for the boundary lying on the domain boundary.

Therefore such boundary condition is necessary to account for the balance of quantities.


\subsection{Periodic boundary conditions}
Periodic boundary condition is always specified on two parts $\Gamma_1, \Gamma_2$ of the domain boundary that share the bijection mapping between points:
$$
\left[x_1, y_1, z_1\right] \leftrightarrow \left[x_2, y_2, z_2\right] \forall \left[x_1, y_1, z_1\right] \in \Gamma_1,\ \forall \left[x_2, y_2, z_2\right] \in \Gamma_2,
$$
so that for each pair of related points $\left[x_1, y_1, z_1\right] \leftrightarrow \left[x_2, y_2, z_2\right]$, the values must be the same:
$$
u\left(\left[x_1, y_1, z_1\right]\right) = u\left(\left[x_2, y_2, z_2\right]\right) \forall \left[x_1, y_1, z_1\right] \in \Gamma_1,\ \forall \left[x_2, y_2, z_2\right] \in \Gamma_2:\, \left[x_1, y_1, z_1\right] \leftrightarrow \left[x_2, y_2, z_2\right].
$$
