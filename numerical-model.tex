\chapter{Numerical model}
The weak formulation of the problem we obtained in \ref{weakSlnDef} still posses a problematic attribute - the space defined in \ref{Bochner} is of infinite dimension, and therefore we would need to employ analytical methods to find the solution \ref{weakSlnDef} in such a space. The equation \ref{WeakFinal} is however rather impossible to be solved analytically, and we have to utilize some sort of numerical simulation - which in turn needs to operate on finite-dimensional spaces. But we need to make sure that the simplifying (reducing) assumptions we make on the way to the numerical model are acceptable so that the numerical solution we obtain converges (as we reduce the discretization size) to the solution defined in \ref{weakSlnDef}.

\paragraph{}
In this chapter we shall consider that $\Omega_t = \Omega\, \forall t \in \lo 0, T\ro $, i.e. the computational domain does not change with respect to time. There are approaches to numerical simulation of MHD phenomena without this condition in place, which utilize the exact same general approach described in this work plus they add additional steps in the algorithm. These are outside of the scope of this work. Also, we always take $\Omega \subset \mathbb{R}^3$.

\section{Triangulation}
\label{section:triangulation}
We start with leaving the time-derivative untouched, and focus on the discretization in space for now - we are performing a \textit{space semidiscretization}.
\paragraph{}
First step in the process of the discretization is to divide the computational domain $\overline{\Omega}$ into a finite number of subsets with properties described below. These subsets form the set, further denoted by $ T_h$, called the \textit{triangulation of the domain $\Omega$}. The parameter $h>0$ of the triangulation usually represents maximum of diameters of all elements $K\in T_h$. The elements $K\in T_h$ are in the context of the finite volume method called $finite\ volumes$.
\\\ \\Properties of $ T_h$:
\begin{enumerate}
    \item Each $K\in T_h$ is closed and connected with its interior $K^{\circ}\neq\emptyset$.
    \item Each $K\in T_h$ has a Lipschitz boundary.
    \item$\cup_{K\in T_h}K\,=\,\overline{\Omega}$
    \item If $K_1,K_2\in T_h$, $K_1\neq{K_2}$, then $K_1^{\circ}\cap{T}_2^{\circ} = \emptyset$.
\end{enumerate}
\paragraph{}
In our case of the three-dimensional problem, we assume that the domain $\Omega$ is obtained as an approximation of the original computational domain (also denoted by $\Omega$), and the triangulation is chosen accordingly to the following attributes:
\renewcommand{\labelenumi}{\Alph{enumi})}
\begin{enumerate}
    \item Each $K\in T_h$ is a closed rectangular parallelepiped, possibly with curved edges.
    \item For $K_1,K_2\in T_h,\,K_1\neq{K}_2$ we have either $K_1\cap{K}_2 = \emptyset$ or $K_1,K_2$ share one edge (if the shared edge is a whole common edge, we call the triangulation \emph{regular}), or $K_1,K_2$ share one vertex, or $K_1,K_2$ share one face.
    \item$\cup_{K\in T_h}K\,=\,\overline{\Omega}.$
\end{enumerate}
Furthermore
\be
\label{Idef}  T_h = \left\{K_i, i\in I\right\},
\ee
where $I\subset Z^+ = \left\{0, 1, 2, ...\right\}$ is a suitable index set.\\
By $\Gamma_{ij}$ we denote a common face between two
neighboring elements $K_i$ and $K_j$. We set 
$$s
\lo i\ro = \left\{j\in I; K_j \text{ is a neighbor of } K_i\right\}.
$$
The boundary $\partial\Omega$ is formed by a finite number of faces of elements $K_i$ adjecent to
$\partial\Omega$. We denote all these boundary faces by $S_j$, where $j\in I_b\subset Z^{-} = \left\{-1, -2, ...\right\}$.
Now we set 
$$
\gamma\lo i \ro = \left\{j\in I_b; S_j \text{ is a face of } K_i\in T_h\right\}
$$ 
and 
$$
\Gamma_{ij} = S_j\text{ for } K_i\in  T_h\text{ such that }S_j\subset\partial K_i, j\in I_b.
$$
For $K_i$ not containing any boundary face $S_j$ we set $\gamma\lo i \ro = \emptyset$.\\
Obviously, $s\lo i \ro \cup\gamma\lo i\ro = \emptyset$ for all $i\in I$. If we write $S\lo i \ro = s \lo i\ro \cup \gamma\lo i \ro$, we have
$$
\partial K_i = \cup_{j\in S\lo i \ro}\Gamma_{ij},\ \ \ \partial K_i\cap\partial{\Omega} = \cup_{j\in\gamma\lo i \ro}\Gamma_{ij}.
$$
Furthermore we define the set of internal (i.e. not lying on the boundary $\partial\Omega$) edges as:
\be
\label{InternalEdges} \Gamma_I = \cup_{i\in I} \cup_{j \notin \gamma\lo i \ro} \Gamma_{ij}
\ee
\paragraph{Note}
If we were to use not $\Omega\subset\mathbb{R}^3$, but rather $\Omega\subset\mathbb{R}^4$, we may just employ the following machinery also to the time-derivative - this is not an uncommon approach. Why the approach described in this work is favored by the author is twofold:
\begin{itemize}
    \item Data (in a general sense - e.g. algebraic systems, function bases, etc.) are smaller when using a separate handling for time-derivative
    \item The dependency on time and space may (and usually does) vary a lot for physical phenomena - to have a separate approach is therefore beneficial
\end{itemize}

\section{Discontinuous Galerkin method}

\subsection{Overview of Discontinuous Galerkin method}
TODO: Z diplomky

\section{Discontinuous Galerkin method application}

\subsection{Resulting linear algebraic structures}

\section{Numerical fluxes}
TODO: Z prace odkud to Honza naimplementoval

\section{Discretization in time}
Relations \ref{DG2} represent a system of ordinary differential equations which can be solved by a suitable numerical method. Since we are interested in applying the Rothe's method (as opposed to the method of lines, which switches the order of discretization in time and space), we now want to discretize the time derivative. In order to do so, we consider a partition $0 = t_0 < t_1 < t_2 < ...$ of the time interval $\lo 0, T\ro$ and set $\tau_k = t_{k+1} - t_k$. We use the notation $\bfw_h^k$ for the approximation of $\bfw_h\lo t_k\ro$.

\subsection{Discrete problem}
\label{section:discreteProblem}
Then we apply the simple implicit \emph{backward Euler method} and our \emph{fully discrete problem} reads: for each $k > 0$ find $\bfw_h^{k+1}$ such that
\begin{enumerate}
    \item ${\mrPsi_h}^{k+1} \in \left[V_h\right]^8$,
    \item For all test functions $\mrvh\in\left[V_h\right]^8$:
    \begin{eqnarray}
    \label{DiscretizedFull} \int_{\Omega_{t}} \frac{{\mrPsi_h}^{k+1} - {\mrPsi_h}^{k}}{\tau} \mrvh & - & \sum_{K_i \in T_h}\int_{K_i}\mrF\lo{\mrPsi_h^{k+1}}\ro \lo\nabla \cdot \mrvh\ro\\ \nonumber & + & \sum_{\Gamma_{ij}\in\Gamma_I} \int_{\Gamma_{ij}} \mrH\lo{\mrPsi_h^{k+1}}|_{ij}, {\mrPsi_h^{k+1}}|_{ji}, \bfn_{ij}\ro \mrvh \\\nonumber
     & + & \sum_{\Gamma_{ij}\in\Gamma_B} \int_{\Gamma_{ij}} \mrH\lo{\mrPsi_h^{k+1}}|_{ij}, \overline{{\mrPsi_h^{k+1}}|_{ji}}, \bfn_{ij}\ro \mrvh\\\nonumber
     & = & \int_{\Omega_{t}} \mrS \mrvh,
    \end{eqnarray}
    \item ${\mrPsi_h}^{0}\lo\bfx\ro = \Pi_h \mrPsi^0\lo\bfx\ro$,
\end{enumerate}
where $\Pi_h$ is a projection of the initial condition $\mrPsi^0$ onto $\left[V_h\right]^8$.
\subsection{Time step length}
\label{section:CFL}
Time step length is an important attribute of the discretization. If it is too small, the calculation might be taking too long to finish, with unnecessary precision with respect to time. If it is too large, we may end up with unstable calculation and obtain results with nonphysical oscillations, or without a solution whatsoever. That is why we need to take extra care to derive the proper value. From the stability perspective, we have a condition for the upper bound of the time step - this condition is called the \textit{Courant-Friedrichs-Lewy} condition. This condition is of the following form:
\be
\tau_{max} = \text{min}\left\{\frac{{\Delta_{x}}_{min}}{v_{max}}, \frac{{\Delta_{x}}_{min}^2}{2 \eta_{max}}\right\},
\ee
where
${\Delta_{x}}_{min}$ is the smallest dimension of any element, $\eta_{max}$ highest resistivity in the domain, and $v_{max}$ highest velocity in the domain, where the following velocities are taken into account:
\begin{eqnarray}
c_s & = & \sqrt{\frac{\gamma\lo\gamma-1\ro}{\rho}\lo U - \rho v^2-U_B\ro},\\
v_A & = & \sqrt{\frac{B^2}{\rho}},\\
v & = & \frac{\sqrt{\pi^2}}{\rho},
\end{eqnarray}
where $c_s$ is the speed of sound, $v_A$ is the Alfv�n speed, and $v$ is the speed of plasma. We then take
$$
v_{max} = \text{max}\left\{c_s, v_A, v \right\}.
$$

This scheme leads to a system of highly nonlinear algebraic equations whose numerical solution is rather complicated. In order to simplify the problem, in the following we shall linearize relations \ref{DiscretizedFull} and obtain a linear system.
\subsection{Linearization}
We need to linearize the two nonlinear terms in \ref{DiscretizedFull}. We shall linearize the first term
$$
\mrF\lo{\mrPsi_h^{k+1}}\ro \lo\nabla \cdot \mrvh\ro
$$
Using the Jacobian matrices $\mrAi$ of the fluxes $\mrFi$
$$
\mrAi\lo{\mrPsi}\ro = \frac{d\mrFi\lo{\mrPsi}\ro}{\mrPsi},\,i = 1, 2, 3
$$
whose existence can be proven, we can linearize this term in the following way:
$$
\mrFi\lo{\mrPsi_h^{k+1}}\ro \lo\nabla \cdot \mrvh\ro\,\rightarrow \mrAi\lo{\mrPsi_h^{k}}\ro \mrPsi_h^{k+1} \lo\nabla \cdot \mrvh\ro,
$$
which is a term linear with respect to $\mrPsi_h^{k+1}$.

\paragraph{}
The second nonlinear term,
$$
\mrH\lo{\mrPsi_h^{k+1}}|_{ij}, {\mrPsi_h^{k+1}}|_{ji}, \bfn_{ij}\ro \mrvh,
$$
where $\mrH$ used in our implementation is the HLLD numerical flux (\citep{hlld}), is much harder to be linearized, and we will handle it by taking the values from the previous time-step:
$$
\mrH\lo{\mrPsi_h^{k+1}}|_{ij}, {\mrPsi_h^{k+1}}|_{ji}, \bfn_{ij}\ro \mrvh\,\approx  \mrH\lo{\mrPsi_h^{k}}|_{ij}, {\mrPsi_h^{k}}|_{ji}, \bfn_{ij}\ro \mrvh.
$$
With this linearization approach, we obtain the following \textit{linearized semi-implicit fully discrete scheme}:
\begin{eqnarray}
\label{DiscretizedLinear} \int_{\Omega_{t}} \frac{{\mrPsi_h}^{k+1} - {\mrPsi_h}^{k}}{\tau} \mrvh & - & \sum_{K_i \in T_h}\int_{K_i}\mrA\lo{\mrPsi_h^{k}}\ro \mrPsi_h^{k+1} \lo\nabla \cdot \mrvh\ro\\\nonumber & = & \int_{\Omega_{t}} \mrS \mrvh\\\nonumber & - &\sum_{\Gamma_{ij}\in\Gamma_I} \int_{\Gamma_{ij}}\mrH\lo{\mrPsi_h^{k}}|_{ij}, {\mrPsi_h^{k}}|_{ji}, \bfn_{ij}\ro \mrvh
\\\nonumber & - &\sum_{\Gamma_{ij}\in\Gamma_B} \int_{\Gamma_{ij}} \mrH\lo{\mrPsi_h^{k}}|_{ij}, \overline{{\mrPsi_h^{k}}|_{ji}}, \bfn_{ij}\ro \mrvh,
\end{eqnarray}
which substitutes \ref{DiscretizedFull} in the definition of discrete problem (\ref{section:discreteProblem} on page \pageref{section:discreteProblem}).

\section{Algebraic formulation}
Last step in the DG method discretization is to transform the system of equations \ref{DiscretizedLinear} into a system of linear algebraic equations at every time step $t_k$ and obtain the solution at this time step as the solution of this linear algebraic system.
\paragraph{}
First, we rearrange the system in the following manner:
\begin{eqnarray}
\label{RewrittenLinearSystem} \sum_{K_i \in T_h}\int_{K_i} \left[\mrvh - \tau\mrA\lo{\mrPsi_h^{k}}\ro \lo\nabla \cdot \mrvh\ro\right] {\mrPsi_h}^{k+1} & = &
\sum_{K_i \in T_h}\int_{K_i} \left[{\mrPsi_h}^{k} + \tau\mathrm{S}\right] \mrvh \\\nonumber& - &\sum_{\Gamma_{ij}\in\Gamma_I} \int_{\Gamma_{ij}}\mrH\lo{\mrPsi_h^{k}}|_{ij}, {\mrPsi_h^{k}}|_{ji}, \bfn_{ij}\ro \mrvh
\\\nonumber& - &\sum_{\Gamma_{ij}\in\Gamma_B} \int_{\Gamma_{ij}} \mrH\lo{\mrPsi_h^{k}}|_{ij}, \overline{{\mrPsi_h^{k}}|_{ji}}, \bfn_{ij}\ro \mrvh.
\end{eqnarray}
Now
\be
\label{Coeffs} {\mrPsi_h}^{k+1} = \sum_{l = 0}^{l = L} y_l {\mrvh}_l, L = \mathrm{dim}\lo\left[V_h\right]^8\ro
\ee
for some (obviously finite) basis $\left\{{v_h}_1, ..., {v_h}_L\right\}$ of $\left[V_h\right]^8$.
Next, since $\mrPsi_h^{k}$, $\tau$, $S$, $\mrA$ (and the basis) are all known, we can define
\begin{eqnarray}
\label{Linear1}
a_{lm} & = & \sum_{K_i \in T_h}\int_{K_i} \left[\mrvhl - \tau\mrA\lo{\mrPsi_h^{k}}\ro \lo\nabla \cdot \mrvhl\ro\right] \mrvhm, \\
b_{l} & = & \sum_{K_i \in T_h}\int_{K_i} \left[{\mrPsi_h}^{k} + \tau\mathrm{S}\right] \mrvhl\\\nonumber & - &\sum_{\Gamma_{ij}\in\Gamma_I} \int_{\Gamma_{ij}}\mrH\lo{\mrPsi_h^{k}}|_{ij}, {\mrPsi_h^{k}}|_{ji}, \bfn_{ij}\ro \mrvhl\\\nonumber& - &
\sum_{\Gamma_{ij}\in\Gamma_B} \int_{\Gamma_{ij}} \mrH\lo{\mrPsi_h^{k}}|_{ij}, \overline{{\mrPsi_h^{k}}|_{ji}}, \bfn_{ij}\ro \mrvh,\\
A & = & \left\{a_{lm}\right\}_{l,m = 1}^{l,m = L},\\
b & = & \left\{b_{l}\right\}_{l = 1}^{l = L},\\
y & = & \left\{y_{l}\right\}_{l = 1}^{l = L},
\label{Linear5}
\end{eqnarray}
and rewriting \ref{RewrittenLinearSystem} using \ref{Linear1} - \ref{Linear5}, we come to the \textit{fully discrete algebraic problem at time instance $t_{k+1}$}:
\be
\label{Alg} Ay = b,
\ee
whose well-posedness, and other attributes that allow for a successful solution of this system, come from the properties of the DG method.
Now if we solve the system \ref{Alg}, and obtain the solution vector $y$, we are able to reconstruct the discrete solution ${\mrPsi_h}^{k+1} \in \left[V_h\right]^8$ using the relation \ref{Coeffs}.

\section{Calculation considerations}

\subsection{Stability}

\subsection{Convergence}

\subsection{Dissipation}