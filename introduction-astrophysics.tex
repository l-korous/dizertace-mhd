\section{Magnetohydrodynamics in Astrophysics}
There are several phenomena in the universe that we can look at as magnetohydrodynamic in nature - planets consisting of metals, interplanetary space, but mainly - stars. If we talk about the nearest star - and the only one we are able to study well enough - the Sun - these phenomena include those that occur in the sun's photosphere (the layer of sun that is visible): sun spots, but also phenomena that occur above the sun (further from the center of the sun): in sun's chromosphere, or even corona (solar flares) - even phenomena that originate from the Sun, but then spread through our solar system - solar winds, space weather. All these phenomena have a large impact on the lives of all of us. For example solar flares (that are often followed by ejection of mass out of the Sun - the so called coronal mass ejections - CMEs) have impact on the Earth's magnetic field which in turn has impact on the electronic communication down on Earth (because the communication satellites used for transmissions may be damaged by the disturbances in the magnetic field). Also people operating at high altitudes, both in airplanes and manned space missions are exposed to the energetic particles coming from the Sun (this term is sometimes called \textit{cosmic rays}). For all the above reasons, it is of great importance to understand the phenomena of space weather, and other MHD phenomena that occur in space.
\subsection{Magnetic reconnection}

Magnetic reconnection occurs within electrically charged gases called plasmas. These charged particles interact strongly with the magnetic field, but at the same time their motions modify the magnetic field. Plasmas behave unlike what we regularly experience on Earth because they travel with their own set of magnetic fields entrapped in the material. Changing magnetic field affects the way charged particles move and vice versa, so the net effect is a complex, constantly-adjusting system that is sensitive to minute variations.

Under normal conditions, the magnetic field lines inside plasmas don't break or merge with other field lines. But sometimes, as field lines get close to each other, the entire pattern changes and everything realign into a new configuration. The amount of energy released can be formidable. Magnetic reconnection taps into the stored energy of the magnetic field, converting it into heat and kinetic energy that sends particles streaming out along the field lines.

Solar flares, which are among the phenomena which are the most important to study, are driven by magnetic reconnection - and thus studying magnetic reconnection is of utmost importance.