\subsection{Magnetohydrodynamics in Astrophysics}
There are several phenomena in the universe that we can look at as Magnetohydrodynamic in nature - planets consisting of metals, interplanetary space, but mainly - stars. If we talk about the nearest star - and the only one we are able to study well enough - the Sun - these phenomena include those occuring in the sun's photosphere (the layer of sun that is visible): sun spots, but also phenomena that occur 'above' the sun: in sun's chromosphere, or even corona (solar flares) - even phenomena that originate from the Sun, but then spread through our solar system - solar winds, space weather. All these phenomena have a large impact on the lives of all of us.