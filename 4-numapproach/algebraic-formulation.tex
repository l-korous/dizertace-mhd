\section{Algebraic formulation}
Last step in the DG method discretization is to transform the system of equations \Cref{DiscretizedLinear} into a system of linear algebraic equations at every time step $t_k$ and obtain the solution at this time step as the solution of this linear algebraic system.
\paragraph{}
First, we rearrange the system in the following manner:
\begin{align}
\label{RewrittenLinearSystem} \sum_{K^i \in T_h}\int_{K^i} \mrvh {\mrPsi_h}^{k+1} & = 
\sum_{K^i \in T_h}\int_{K^i} \left[{\mrPsi_h}^{k} + \tau\mrS + \tau\mrA\lo{\mrPsi_h^{k}}\ro \lo\nabla \cdot \mrvh\ro\right] \mrvh \\\nonumber& - \sum_{\Gamma_{ij}\in\Gamma_I} \int_{\Gamma_{ij}}\mrH\lo{\mrPsi_h^{k}}|_{ij}, {\mrPsi_h^{k}}|_{ji}, \bfn_{ij}\ro \mrvh
\\\nonumber& - \sum_{\Gamma_{ij}\in\Gamma_B} \int_{\Gamma_{ij}} \mrH\lo{\mrPsi_h^{k}}|_{ij}, \overline{{\mrPsi_h^{k}}|_{ji}}, \bfn_{ij}\ro \mrvh.
\end{align}
We can see that the left hand side does not depend on the previous solution values, so there is no need to recalculate the matrix entries in every time step (unless we employ AMR, in which case the mesh and therefore the set of basis functions changes).
Now
\be
\label{Coeffs} {\mrPsi_h}^{k+1} = \sum_{l = 0}^{l = L} y_l {\mrvh}_l, L = \mathrm{dim}\lo\left[V_h\right]^8\ro
\ee
for some (obviously finite) basis $\left\{{v_h}_1, ..., {v_h}_L\right\}$ of $\left[V_h\right]^8$.
Next, since $\mrPsi_h^{k}$, $\tau$, $S$, $\mrA$ (and the basis) are all known, we can define
\begin{align}
\label{Linear1}
a_{lm} & =  \sum_{K^i \in T_h}\int_{K^i} \mrvhl \mrvhm, \\
\label{Linear2}
b_{l} & =  \sum_{K^i \in T_h}\int_{K^i} \left[{\mrPsi_h}^{k} + \tau\mrS + \tau\mrA\lo{\mrPsi_h^{k}}\ro \lo\nabla \cdot \mrvhl\ro\right] \mrvhl\\\nonumber & - \sum_{\Gamma_{ij}\in\Gamma_I} \int_{\Gamma_{ij}}\mrH\lo{\mrPsi_h^{k}}|_{ij}, {\mrPsi_h^{k}}|_{ji}, \bfn_{ij}\ro \mrvhl\\\nonumber& - 
\sum_{\Gamma_{ij}\in\Gamma_B} \int_{\Gamma_{ij}} \mrH\lo{\mrPsi_h^{k}}|_{ij}, \overline{{\mrPsi_h^{k}}|_{ji}}, \bfn_{ij}\ro \mrvhl,\\
\label{Linear3}
A & =  \left\{a_{lm}\right\}_{l,m = 1}^{l,m = L},\\
\label{Linear4}
b & =  \left\{b_{l}\right\}_{l = 1}^{l = L},\\
y & =  \left\{y_{l}\right\}_{l = 1}^{l = L},
\label{Linear5}
\end{align}
and rewriting \Cref{RewrittenLinearSystem} using \Cref{Linear1} - \Cref{Linear5}, we come to the \textit{fully discrete algebraic problem at time instance $t_{k+1}$}:
\be
\label{Alg} Ay = b,
\ee
whose well-posedness, and other attributes that allow for a successful solution of this system, come from the properties of the DG method.
Now if we solve the system \Cref{Alg}, and obtain the solution vector $y$, we are able to reconstruct the discrete solution ${\mrPsi_h}^{k+1} \in \left[V_h\right]^8$ using the relation \Cref{Coeffs}.