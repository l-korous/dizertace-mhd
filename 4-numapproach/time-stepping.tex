\section{Discretization in time}
Relations \Cref{DG2} represent a system of ordinary differential equations which can be solved by a suitable numerical method. Since we are interested in applying the Rothe's method (as opposed to the method of lines, which switches the order of discretization in time and space), we now want to discretize the time derivative. In order to do so, we consider a partition $0 = t_0 < t_1 < t_2 < ...$ of the time interval $\lo 0, T\ro$ and set $\tau_k = t_{k+1} - t_k$. We use the notation $\bfw_h^k$ for the approximation of $\bfw_h\lo t_k\ro$.

\subsection{Discrete problem}
\label{section:discreteProblem}
Then we apply a time discretization scheme, for example, the simple explicit \emph{Euler method} and our \emph{fully discrete problem} reads: for each $k > 0$ find $\bfw_h^{k+1}$ such that
\begin{enumerate}
    \item ${\mrPsi_h}^{k+1} \in \left[V_h\right]^8$,
    \item For all test functions $\mrvh\in\left[V_h\right]^8$:
			\begin{align}
			\label{DiscretizedFull} \int_{\Omega_{t}} \frac{{\mrPsi_h}^{k+1} - {\mrPsi_h}^{k}}{\tau} \mrvh & - \sum_{K^i \in T_h}\int_{K^i}\mrF\lo{\mrPsi_h^{k}}\ro \lo\nabla \cdot \mrvh\ro\\
			\nonumber & + \sum_{\Gamma_{ij}\in\Gamma_I} \int_{\Gamma_{ij}} \mrH\lo{\mrPsi_h^{k}}|_{ij}, {\mrPsi_h^{k}}|_{ji}, \bfn_{ij}\ro \mrvh \\
			\nonumber & + \sum_{\Gamma_{ij}\in\Gamma_B} \int_{\Gamma_{ij}} \mrH\lo{\mrPsi_h^{k}}|_{ij}, \overline{{\mrPsi_h^{k}}|_{ji}}, \bfn_{ij}\ro \mrvh\\
			\nonumber & = \int_{\Omega_{t}} \mrS \mrvh,
			\end{align}
    \item ${\mrPsi_h}^{0}\lo\bfx\ro = \Pi_h \mrPsi^0\lo\bfx\ro$,
\end{enumerate}
where $\Pi_h$ is a projection of the initial condition $\mrPsi^0$ onto $\left[V_h\right]^8$.
\subsection{Time step length}
\label{section:CFL}
Time step length is an important attribute of the discretization. If it is too small, the calculation might be taking too long to finish, with unnecessary precision with respect to time. If it is too large, we may end up with unstable calculation and obtain results with nonphysical oscillations, or without a solution whatsoever. That is why we need to take extra care to derive the proper value. From the stability perspective, we have a condition for the upper bound of the time step - this condition is called the \textit{Courant-Friedrichs-Lewy} condition. This condition is of the following form:
\be
\label{equation:CFLcond}
\tau_{max} = \text{min}\left\{\frac{{\Delta_{x}}_{min}}{c_{max}}, \frac{{\Delta_{x}}_{min}^2}{2 \eta_{max}}\right\},
\ee
where
${\Delta_{x}}_{min}$ is the smallest dimension of any element, $\eta_{max}$ highest resistivity in the domain, and $v_{max}$ highest velocity in the domain, where the following velocities are taken into account:
\begin{align}
c_s & =  \sqrt{\frac{\gamma\lo\gamma-1\ro}{\rho}\lo U - \rho v^2-U_B\ro},\\
c_a & =  \sqrt{\frac{B^2}{\rho}},\\
u,
\end{align}
where $c_s$ is the speed of sound, $c_a$ is the Alfv�n speed, and $u$ is the speed of plasma. We then take
$$
c_{max} = \text{max}\left\{c_s, c_a, u \right\}.
$$