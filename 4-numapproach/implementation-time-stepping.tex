\section{Time-stepping and linearization}

\subsection{Time-stepping}
The simple time discretization scheme that we use in \cref{section:discreteProblem} allows us to simply implement the time-stepping in the following fashion:\\
\begin{algorithm}[H]
\textbf{    Set: }$y_0 =\ $ (initial solution)\\
\textbf{    Set: }$ts = 1 $ \# initial time step\\
\textbf{    Set: }$t = 0.00 $ \# initial time\\
    \# Loop over time steps\\
    \For{$;\ t < T;\ t = t+\tau,\ ts = ts + 1$}{
        \KwData{Solution from the previous time step $y^{ts - 1}$}
1.Call procedure \cref{algorithm:singleTimeStep} to obtain $A, b\lo y^{ts - 1}\ro$\\
2.Solve the problem $Ay^{ts} = b\lo y^{ts - 1}\ro\ $ \# See note below\\
3.(If necessary) postprocess $y^{ts}$ using \cref{algorithm:limiter}\\
4.Calculate updated value of $\tau$ using \cref{equation:CFLcond}\\
	 }
    \caption{Time-stepping procedure}
\label{algorithm:timeStepping}
\end{algorithm}
\paragraph{Note}
\label{note:solvers}
The step 2 (solving the algebraic problem) is of course a key point in the overall process. Because of its importance, the aim of this work is not to describe, or even implement an algebraic solver for this purpose. Many scientific teams have spent many years on publicly available open-source solvers that are usable by the software we develop for the purpose of solving the MHD phenomena. We use the existing solvers.