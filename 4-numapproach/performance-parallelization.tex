\subsection{Parallelization}
\label{section:parallel}
To be able to perform any large scale calculations, we need to be able to utilize the performance of hardware at maximum. Being able to use modern, multi-core computers is an absolute must to achieve good performance, as the execution time when using parallel execution can decrease by a factor of corresponding to the number of cores - and modern machines have tens of cores available.

The parallelization is possible at several places in the overall algorithm \ref{algorithm:singleTimeStep}. But it makes most sense to parallelize the outer-most loop over elements, and over edges.

\paragraph{}
Another point for parallelization is the algebraic solver. As explained in \ref{note:solvers}, we rely on existing software packages for finding the solution of the algebraic system \ref{Alg} - all the used solvers support and heavily utilize parallelization.