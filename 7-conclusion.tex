\chapter{Conclusion, outlook}
As for the mathematical, numerical, and technical (software) parts for such tool to be delivered, all problems that were to be solved, such as
\begin{itemize}
	\item Shock-capturing for high-order DG scheme to prevent non-physical oscillations (through Vertex-based limiter, see \cref{algorithm:limiter}),
	\item Adaptive algorithm for the discretization of the space derivatives (through AMR, entire \cref{chapter:amr}),
	\item A specific shapeset of basis and test functions based on Taylor expansions (through \cref{sec:divFreeSpace}),
	\item Adaptive algorithm for the discretization of the time derivative (through CFL condition, see \cref{equation:CFLcond}),
\end{itemize}
have been solved, and moreover performance level meets the needs of the use cases. All this has been shown on benchmark problems (see \cref{sec:benchmarks}), as well as real-world Titov-Demoulin-based simulation.
\textit{}
Of course, much can be improved upon, for example:
\begin{itemize}
	\item Second-order scheme for the discretization of the time derivative,
	\item Adaptive algorithm for the discretization of the time derivative,
	\item Caching of values that are necessary in multiple spaces of the algorithm (utilizing the RAM),
	\item Further use of vectorization for evaluation of integral quantities,
\end{itemize}
but the original goal of preparing an easy-to-use, easy-to-extend, and well programmed, and tested software package, has been achieved:
\begin{itemize}
\item The code is able to utilize large-scale clusters through implementation being based on Message Passing Interface (MPI),
\item The code is publicly available, and well documented - by following the schematics of the used numerical methods, as well as using clear naming conventions in the object model of the program,
\item The code is quite ready for addition of new tests, new benchmarks, new examples, as well as easy parametrization of the existing ones.
\end{itemize}

\section*{Outloook}
Further work will focus on real-world astrophysical problems, where the Titov-Demoulin-based simulations, although all mathematical and numerical apparatus is in place, still need work to satisfy the goal of being reliable and all-purpose tool for astrophysicists. Further work will focus on incorporating additional relevant physical phenomena - mainly study of the magnetic field reconnection - \citep{reconnection}, and other phenomena occurring both in solar physics and in industrial applications of plasma flow.

To sum up next steps with the already finished toolset, these are the logical next steps:
\begin{itemize}
\item Replicate fully the results of \cite{miraClanek}, including all parameters, and boundary condition specifics,
\item Run much more detailed simulation over a larger domain for a longer time, to be able to inspect the destructive behavior of the astrophysical event on small scales,
\item Continuously improve the performance of the code, fix issues as they are discovered, and extend the implemented set of numerical schemes,
\item Add additional relevant physics phenomena - resistivity, magnetic field reconnection, possibly relativistic effects,
\item Get in touch with other possible users of the implemented software to enrich the set of possible use cases.
\end{itemize}

\newpage
\section*{Own publications}
This Section lists all publications of the author.
\subsection*{Journals with Impact Factor}
\begin{enumerate}[label={[\arabic*]}]
\item
Vadym Aizinger, Dmitri Kuzmin, Lukas Korous, \textit{Scale separation in fast hierarchical solvers for discontinuous Galerkin methods}, September 2015, Applied Mathematics and Computation 266:838-849, DOI: 10.1016/j.amc.2015.05.047

\item
Lukas Korous, Pavel Solin, \textit{An adaptive hp-DG method with dynamically-changing meshes for non-stationary compressible Euler equations}, May 2012, COMPUTING, 95,1,425-444, DOI: 10.1007/s00607-012-0257-1

\item
Pavel Solin, Lukas Korous, Pavel Kus, \textit{Hermes2D, a C++ library for rapid development of adaptive hp-FEM and hp-DG solvers}, November 2014, Journal of Computational and Applied Mathematics 270:152-165, DOI: 10.1016/j.cam.2014.02.007

\item
Pavel Solin, Ondrej Certik, Lukas Korous, \textit{Three anisotropic benchmark problems for adaptive finite element methods}, March 2013, Applied Mathematics and Computation 219(13), DOI: 10.1016/j.amc.2010.12.080

\item
Zhonghua Ma, Lukas Korous, Erick Santiago, \textit{Solving a suite of NIST benchmark problems for adaptive FEM with the Hermes library}, December 2012, Journal of Computational and Applied Mathematics 236(18):4846-4861, DOI: 10.1016/j.cam.2012.02.004

\item
Pavel Solin, Lukas Korous, \textit{Space-time adaptive hp-FEM for problems with traveling sharp fronts}, May 2012, Computing 95(1), DOI: 10.1007/s00607-012-0243-7

\item
Pavel Solin, Lukas Korous, Adaptive higher-order finite element methods for transient PDE problems based on embedded higher-order implicit Runge-Kutta methods, February 2012, Journal of Computational Physics 231(4):1635-1649, DOI: 10.1016/j.jcp.2011.10.023
\end{enumerate}

\subsection*{Conference papers}

\begin{enumerate}[label={[\arabic*]}]
\item 
Lukas Korous, Pavel Karban, \textit{Distributed Implicit Discontinuous Galerkin MHD Solver}, COMPUMAG 2017, Daejeon, Korea
	
\item
Ivo Dolezel, Lukas Korous, Pavel Karban, Frantisek Mach, \textit{Higher-Order Eggshell Method for Computation of Forces Acting on Ferromagnetic Bodies}, January 2014, Conference: 9th IET International Conference on Computation in Electromagnetics (CEM 2014), DOI: 10.1049/cp.2014.0188

\item
Korous Lukas, \textit{Adaptive hp-DG Method for Nonstationary Compressible Euler Equations}, Elektrotechnika a informatika 2012. Part 1., Elektrotechnika,53-56
\end{enumerate}

\subsection*{Software}

\begin{enumerate}[label={[\arabic*]}]
\item
Lukas Korous, Pavel Solin, Milan Hanus, Jakub Cerveny, \textit{The library for physical fields modelling using hp-adaptive finite element method Hermes2D}

\item
Lukas Korous, Pavel Solin, Pavel Kus, David Andr, \textit{Library for physical fields modelling in three dimensional space using hp-adaptive FEM Hermes3D}
\end{enumerate}