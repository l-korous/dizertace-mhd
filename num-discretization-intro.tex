\section{Triangulation}
\label{section:triangulation}
We start with leaving the time-derivative untouched, and focus on the discretization in space for now - we are performing a \textit{space semidiscretization}.
\paragraph{}
First step in the process of the discretization is to divide the computational domain $\overline{\Omega}$ into a finite number of subsets with properties described below. These subsets form the set, further denoted by $ T_h$, called the \textit{triangulation of the domain $\Omega$}. The parameter $h>0$ of the triangulation usually represents maximum of diameters of all elements $K\in T_h$. The elements $K\in T_h$ are in the context of the finite volume method called $finite\ volumes$.
\\\ \\Properties of $ T_h$:
\begin{enumerate}
    \item Each $K\in T_h$ is closed and connected with its interior $K^{\circ}\neq\emptyset$.
    \item Each $K\in T_h$ has a Lipschitz boundary.
    \item$\cup_{K\in T_h}K\,=\,\overline{\Omega}$
    \item If $K_1,K_2\in T_h$, $K_1\neq{K_2}$, then $K_1^{\circ}\cap{T}_2^{\circ} = \emptyset$.
\end{enumerate}
\paragraph{}
In our case of the three-dimensional problem, we assume that the domain $\Omega$ is obtained as an approximation of the original computational domain (also denoted by $\Omega$), and the triangulation is chosen accordingly to the following attributes:
\renewcommand{\labelenumi}{\Alph{enumi})}
\begin{enumerate}
    \item Each $K\in T_h$ is a closed rectangular parallelepiped, possibly with curved edges.
    \item For $K_1,K_2\in T_h,\,K_1\neq{K}_2$ we have either $K_1\cap{K}_2 = \emptyset$ or $K_1,K_2$ share one edge (if the shared edge is a whole common edge, we call the triangulation \emph{regular}), or $K_1,K_2$ share one vertex, or $K_1,K_2$ share one face.
    \item$\cup_{K\in T_h}K\,=\,\overline{\Omega}.$
\end{enumerate}
Furthermore
\be
\label{Idef}  T_h = \left\{K_i, i\in I\right\},
\ee
where $I\subset Z^+ = \left\{0, 1, 2, ...\right\}$ is a suitable index set.\\
By $\Gamma_{ij}$ we denote a common face between two
neighboring elements $K_i$ and $K_j$. We set 
$$s
\lo i\ro = \left\{j\in I; K_j \text{ is a neighbor of } K_i\right\}.
$$
The boundary $\partial\Omega$ is formed by a finite number of faces of elements $K_i$ adjecent to
$\partial\Omega$. We denote all these boundary faces by $S_j$, where $j\in I_b\subset Z^{-} = \left\{-1, -2, ...\right\}$.
Now we set 
$$
\gamma\lo i \ro = \left\{j\in I_b; S_j \text{ is a face of } K_i\in T_h\right\}
$$ 
and 
$$
\Gamma_{ij} = S_j\text{ for } K_i\in  T_h\text{ such that }S_j\subset\partial K_i, j\in I_b.
$$
For $K_i$ not containing any boundary face $S_j$ we set $\gamma\lo i \ro = \emptyset$.\\
Obviously, $s\lo i \ro \cup\gamma\lo i\ro = \emptyset$ for all $i\in I$. If we write $S\lo i \ro = s \lo i\ro \cup \gamma\lo i \ro$, we have
$$
\partial K_i = \cup_{j\in S\lo i \ro}\Gamma_{ij},\ \ \ \partial K_i\cap\partial{\Omega} = \cup_{j\in\gamma\lo i \ro}\Gamma_{ij}.
$$
Furthermore we define the set of internal (i.e. not lying on the boundary $\partial\Omega$) edges as:
\be
\label{InternalEdges} \Gamma_I = \cup_{i\in I} \cup_{j \notin \gamma\lo i \ro} \Gamma_{ij}
\ee
\paragraph{Note}
If we were to use not $\Omega\subset\mathbb{R}^3$, but rather $\Omega\subset\mathbb{R}^4$, we may just employ the following machinery also to the time-derivative - this is not an uncommon approach. Why the approach described in this work is favored by the author is twofold:
\begin{itemize}
    \item Data (in a general sense - e.g. algebraic systems, function bases, etc.) are smaller when using a separate handling for time-derivative
    \item The dependency on time and space may (and usually does) vary a lot for physical phenomena - to have a separate approach is therefore beneficial
\end{itemize}