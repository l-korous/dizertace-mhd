\subsection{Notation}
In the entire work, we use the notation specified in \ref{table:notation} on page \pageref{table:notation}.

\begin{table}
    \centering
    \begin{tabular}{ |c|c| } 
        \hline
        Symbol & Meaning \\ 
        \hline
        $a$ & a scalar quantity "a"\\
        $\bfa$ & a 3-element vector "a"\\
        $a_i$ & the i-th component of a vector $\bfa$ \\
        \hline
        $\rho$ & Density \\ 
        $\bfpi$ & Momentum \\ 
        $\mu$ & Permeability \\ 
        $\mu_0$ & Permeability of vacuum\\ 
        $t$ & Time variable\\ 
        $\bfx$ & Space variable \\ 
        $p$ & Pressure \\ 
        $\bfu$ & Velocity \\ 
        $u$ & $|\bfu|$ \\ 
        $\bfB$ & Magnetic field \\ 
        $\bfJ$ & Current density, $\bfJ = \nabla \times \bfB$ \\ 
        $B$ & $|\bfB|$ \\ 
        $\bfE$ & Electric field\\ 
        $e$ & Internal energy density \\ 
        $c$ & Speed of light\\ 
        $\bfg$ & Gravitational acceleration\\ 
        $\sigma$ & Conductivity, $\sigma = \frac{1}{\eta}$\\ 
        $\eta$ & Electrical resistivity, $\eta = \frac{1}{\sigma}$\\
        $c_v$ & Specific heat at constant volume\\
        $c_p$ & Specific heat at constant pressure\\
        $\gamma$ & Poisson adiabatic constant\\
        $\theta$ & Absolute temperature\\
        $\bfq$ & Heat flux\\
        $q$ & Electric charge\\
        $\bff$ & Density of force acting on fluid\\
        $\bfF_L$ & Lorentz force, $\bfF_L = q\lo\bfE + \bfu\times\bfB\ro$\\
        $\bff_L$ & Lorentz force density, $\bff_L = q\bfE + \bfJ\times\bfB$\\
        $\epsilon$ & Electrical permittivity\\
        $\epsilon_0$ & Electrical permittivity of vacuum\\
        $\rho_q$ & Charge density\\
        $U_k$ & Kinetic energy, $U_k = \rho\frac{u^2}{2}$ \\ 
        $U_m$ & Magnetic energy, $U_m = \frac{B^2}{2{}\mu_0}$ \\ 
        $U$ & Total energy, $U = U_k + U_m + \rho e$ \\ 
        $\tilde{U}$ & Hydrodynamic energy, $\tilde{U} = U - U_m$ \\ 
        \hline
    \end{tabular}
    \caption{Notation}
    \label{table:notation}
\end{table}

\subsection{Initial setup}

We consider a time interval $\left(0,T\right)$ and space domain $\Omega_{\textit{t}}\subset \mathbb{R}^3$ occupied by a fluid at time $t$.
By $\mathcal{M}$ we denote the space-time domain in consideration: 
\begin{equation}\label{M}
\mathcal{M}=\left\{\left(\textbf{\textit{x}},t\right);\\\textbf{\textit{x}}\in\Omega_{\textit{t}},t\in\left(0,T\right)\right \}.
\end{equation}
Moreover we assume that $\mathcal{M}$ is an open set.

\subsubsection{Assumptions}
When dealing with MHD phenomena in plasma, the following rules apply
\begin{itemize}
    \item We assume that the fluid is inviscid.
    \item We assume that the fluid is compressible.
    \item We consider only the so-called \textit{perfect gas} or \textit{ideal gas} whose state variables satisfy the following \textit{equation of state}
    \begin{equation}\label{start_therm}
    p = R\theta\rho,
    \end{equation}
    where $R$ is the \textit{gas constant}, which is defined as 
    \begin{equation}
    R = c_p - c_v.
    \end{equation} 
\end{itemize}

\subsection{Euler's equations of compressible flow}
This system of equation reads

\begin{eqnarray}
\label{ContinuityEq} \pds{\rho}{t} + \nabla\cdot\left(\bfpi\right) & = & 0\\
\label{NSEq} \pds{\bfpi}{t} + \nabla\cdot\left(\bfpi\otimes\bfu\right)& = & \rho\bff - \nabla\,p,\\
\label{EnergyEq} \pds{\tilde{U}}{t} + \nabla\cdot\lo{\tilde{U}}\bfu\ro & = & \rho\bff\cdot\bfu \,-\,\nabla\cdot\lo{p}\bfu\ro + \nabla\cdot\bfq,
\end{eqnarray}
where $\otimes$ denotes the \textit{tensor product}:
\begin{displaymath}
\bs{a}\otimes\bs{b} =
\left(
\begin{array}{ccc}
a_1b_1 & a_1b_2 & a_1b_3 \\
a_2b_1 & a_2b_2 & a_2b_3 \\
a_3b_1 & a_3b_2 & a_3b_3
\end{array}
\right).
\end{displaymath}

Moreover, the following relations hold:
\begin{eqnarray}
\tilde{U} & = & \rho e + U_k,\\
p & = & \lo{}\gamma-1\ro\lo{\tilde{U}}-U_k\ro, \label{therm_1}\\
\theta & = & \lo{\tilde{U}}/{\rho}-\left|\bfu\right|^2/2\ro/{c_v}.
\end{eqnarray}

This system is simply called the \textit{compressible Euler equations} for a heat-conductive perfect gas. The individual equations are called the \textit{continuity equation}(\ref{ContinuityEq}), the \textit{Navier-Stokes equations}(\ref{NSEq}), and the \textit{energy equation}(\ref{EnergyEq}).

For the force density $f$ we assume that only the Lorentz force and gravity act upon the fluid:
$$
\label{ForceEq} \bff = \bff_L + \bfg.
$$


\subsection{Maxwell's equations of electromagnetism}
This system of equation reads
\begin{eqnarray}
\label{Faraday} \nabla \times \bfE & = & -\pds{\bfB}{t}\\
\label{Ampere} \nabla \times \bfB & = & \mu_0 \lo \bfJ + \epsilon_0 \pds{\bfE}{t}\ro\\\
\label{GaussMag} \nabla \cdot \bfB & = &0\\
\label{Gauss} \nabla \cdot \bfE & = & \frac{\rho_q}{\epsilon_0}.
\end{eqnarray}
The individual equations are known as Faraday's law(\ref{Faraday}), Ampere's law(\ref{Ampere}), and Gauss's laws(\ref{Gauss}, \ref{GaussMag}).


\subsection{Derived relations between electromagnetic quantities}
Further relations that are useful when deriving the MHD equations are:
\begin{eqnarray}
\label{U_mEq} \frac{d}{dt} U_m & = & \frac{1}{\mu_0}\nabla \cdot \lo \bfB\times\bfE\ro - \bfE\cdot\bfJ\\
\label{Ohm} \bfE & = & -u \times \bfB + \frac{\eta}{\mu_0}\bfJ\\
\label{InductionEq} \pds{\bfB}{t} + \nabla \times \lo \bfB \times \bfu \ro & = & -\frac{1}{\mu_0} \nabla \times \lo \eta \nabla \times \bfB \ro
\end{eqnarray}
The equation \ref{Ohm} is called the \textit{Ohm's law}, the equation \ref{InductionEq} the \textit{Induction equation}.

Applying now \ref{ForceEq} to \ref{NSEq}, \ref{EnergyEq}, we obtain:
\begin{eqnarray}
\label{NSEq1} \pds{\bfpi}{t} + \nabla\cdot\left(\bfpi\otimes\bfu\right)& = & q\bfE + \bfJ\times\bfB + \rho\bfg - \nabla\,p,\\
\label{EnergyEq1} \pds{\tilde{U}}{t} + \nabla\cdot\lo{\tilde{U}}\bfu\ro & = & q\bfE\cdot\bfu + \bfJ\times\bfB\cdot\bfu + \rho\bfg\cdot\bfu \,-\,\nabla\cdot\lo{p}\bfu\ro + \nabla\cdot\bfq.
\end{eqnarray}
The adjusted energy equation \ref{EnergyEq1} does not include the magnetic energy $U_m$, which we do want to include in the MHD equations. To achieve this, we employ \ref{U_mEq} - \ref{InductionEq} and rearrange and we obtain the final energy equation
\be
\label{EnergyEqPrefinal} \pds{U}{t} = -\nabla\cdot\left[\bfpi\lo \frac{u^2}{2} + e\ro + p \bfu - \frac{1}{\mu_0}\bfB\times\bfE\right] + \rho \bfg \cdot \bfu + \nabla\cdot\bfq.
\ee

\subsection{Simplifying assumptions}
\subsubsection{Negligible time derivative of Electric field}
For the time increments that we are concerned with, the time derivative in \ref{Ampere} is
very small. To estimate the minimum time increment value $\tau$ which would allow us to neglect the derivative, take the ratio of the two terms on the right hand side of \ref{Ampere}:
\begin{equation}
\epsilon_0 \frac{\pds{\bfE}{t}}{\bfJ} \approx \frac{\frac{\epsilon_0 \bfE}{\tau}}{\sigma\bfE} \approx \frac{\epsilon_0}{\sigma \tau} \approx \frac{10^{-11}}{\tau}.
\end{equation}
This means for time scales much greater than $10^{-11}$ seconds, the time derivative of $\bfE$ can be neglected. As a consequence Equation \ref{Ampere} can be written as:
\begin{equation}
\label{Assumption1} \nabla \times \bfB = \mu_0 \bfJ.
\end{equation}
Using \ref{Assumption1}, we can write
\begin{equation}
\label{Asssumption11} \mu_0 \bfJ \times \bfB = \lo\bfB\cdot\nabla\ro\bfB - \nabla\frac{B^2}{2} = \nabla\cdot\lo\bfB\bfB\ro - \nabla\frac{B^2}{2},
\end{equation}
where the last equality comes from \ref{GaussMag}. And using \ref{Asssumption11} we can rewrite \ref{NSEq1} as
\be
\label{NSEq2} \pds{\bfpi}{t} + \nabla\cdot\left(\bfpi\otimes\bfu\right) =  q\bfE + \nabla\cdot\lo\frac{1}{\mu_0}\bfB\bfB - \frac{B^2}{2}\mathrm{I}\ro + \rho\bfg - \nabla\,p
\ee

\subsubsection{Negligible Electric field in the Navier-Stokes equations}
The magnitude of electric field is smaller than the magnetic field by the factor $\frac{u^2}{c^2}$, so we can neglect the term $q\bfE$ on the right-hand-side of \ref{NSEq2}.

\subsubsection{Negligible heat fluxes}
Since the heat transfer accounts for a negligible contribution to the overall energy transfer, we neglect the heat flux terms, i.e. we set $\bfq = \mathbf{0}$ in the energy equation \ref{EnergyEqPrefinal}.

\subsection{Adding the induction equation}
The induction equation \ref{InductionEq} can be rearranged in the following way:
\begin{eqnarray}
\pds{\bfB}{t} + \nabla \times \lo \bfB \times \bfu \ro & = & -\frac{1}{\mu_0} \nabla \times \lo \eta \nabla \times \bfB \ro\\
\label{InductionPreFinal} \pds{\bfB}{t} & = & - \nabla \times \lo \bfu \otimes \bfB - \bfB \otimes \bfu \ro + \frac{1}{\mu_{0}\sigma} \lo \nabla^2 \bfB \ro.
\end{eqnarray}
Now we can form the system of MHD equations:
\begin{eqnarray}
\label{ContinuityEqFinal} \pds{\rho}{t} & = & - \nabla\cdot\left(\bfpi\right),\\
\label{NSEqFinal} \pds{\bfpi}{t} & = & - \nabla\cdot\left(\bfpi\otimes\bfu\right) + \nabla\cdot\lo\frac{1}{\mu_0}\bfB\bfB - \frac{B^2}{2}\mathrm{I}\ro + \rho\bfg - \nabla\,p,\\
\label{EnergyEqFinal} \pds{U}{t} & = & -\nabla\cdot\left[\bfpi\lo \frac{u^2}{2} + e\ro + p \bfu - \frac{1}{\mu_0}\bfB\times\bfE\right] + \rho \bfg \cdot \bfu\\
\label{InductionFinal} \pds{\bfB}{t} & = & - \nabla \times \lo \bfu \otimes \bfB - \bfB \otimes \bfu \ro + \frac{1}{\mu_{0}\sigma} \lo \nabla^2 \bfB \ro.
\end{eqnarray}
This form suggests that rewriting these equations into a more suitable (for numerical calculations) conservative form shall be possible.
\subsection{Conservative form of the MHD equations}
A conservative form of a system of equations generally takes the form of
\be
\label{conservativeGeneric} \pds{\mathrm{\Psi}}{t} + \nabla \cdot \mathrm{F}\lo\mathrm{\Psi}\ro = \mathrm{S},
\ee
where $\mathrm{\Psi}$ is the so-called \textit{state vector}, $\mathrm{F}_i,\ i = 1, 2, 3$ are the so-called \textit{fluxes}, and $\mathrm{S}$ is the so-called \textit{source term}.
\paragraph{}
Rewriting the system of equations \ref{ContinuityEqFinal} - \ref{InductionFinal} to the form \ref{conservativeGeneric} is fairly straightforward.
We obtain the following:
\begin{eqnarray}
\mathrm{\Psi} & = & \lo\begin{array}{c}\rho \\ \pi_1 \\ \pi_2 \\ \pi_3 \\ U \\ B_1 \\ B_2 \\ B_3 \\ \end{array}\ro,\\\mathrm{F}_i & = & \lo\begin{array}{c} \pi_i \\ \frac{\pi_1 \pi_i}{\rho} - B_1 B_i + \frac12 \delta_{1i} \lo p + U_m\ro \\ \frac{\pi_2 \pi_i}{\rho} - B_1 B_i + \frac12 \delta_{2i} \lo p + U_m\ro \\ \frac{\pi_3 \pi_i}{\rho} - B_1 B_i + \frac12 \delta_{3i} \lo p + U_m\ro \\ \frac{\pi_i}{\rho} \lo \frac{\gamma}{\gamma - 1} p + U_k\ro + 2\eta\epsilon_{ijk} J_j B_k + \frac{2}{\rho} \epsilon_{ijk} \lo \pi_k B_i - \pi_i B_k\ro B_k  \\ \frac{\pi_i B_1 - \pi_1 B_i}{\rho} + \eta \epsilon_{1ij} J_j \\ \frac{\pi_i B_2 - \pi_2 B_i}{\rho} + \eta \epsilon_{2ij} J_j \\ \frac{\pi_i B_3 - \pi_3 B_i}{\rho} + \eta \epsilon_{3ij} J_j \\ \end{array}\ro,\\
\mathrm{S} & = & \lo\begin{array}{c}0 \\ \rho g_1 \\ \rho g_2 \\ \rho g_3 \\ \bfpi \cdot \bfg \\ 0 \\ 0 \\ 0 \\ \end{array}\ro,
\end{eqnarray}
where $J_j = \lo\nabla\times\bfB\ro_j$.