\section{Derivation of the differential model}
From moments of Boltzmann equation we got a two-fluid (ions and electron fluids) continuity equation, motion equation and energy equation. For simplification,
we do not consider emission of radiative looses or the energy sources and the energy dissipation due to particle collisions. The coronal plasma has a
temperature about $\sim1\,MK$ and low density $\sim10^{15}\,\mathrm{m^{-3}}$ due to the collisions are very rare. The equations for ion and electron components
can be summed and we obtain an one-fluid resistivity MHD equations. The one fluid variables are defined as follows
\begin{eqnarray}
 \rho         &=& n_em_e + n_im_i\;, \\
 \rho \vct{u} &=& n_e m_e \vct u_e + n_i m_i \vct u_i \;,\\
 \vct j       &=& ?e n_e \vct u_e + e n_i\vct u_i \;,\\
 p            &=& p_e + p_i = (n_e + n_i) k_\mathrm{B} T \;.
\end{eqnarray}
With one fluid variables we can write MHD equations
\begin{eqnarray}
 \pard{\rho}{t}+\nabla\cdot\lp\rho\vct{u}\rp &=& 0 \,,\label{EqCont}\\
 \rho\lp\pard{\vct{u}}{t}+\vct{u}\cdot\nabla\vct{u}\rp &=& \vct{j}\times\vct{B}-\nabla p+\rho \vct{g} \,,\label{EqMot}\\
 \vct{E}+\vct{u}\times\vct{B} &=& \eta\vct{j}  \,,\label{EqOhm}\\
 %\frac{T\dd s}{\dd t}=\der{q}{t} &=& \eta\vct{j}^2 +\rho\vct{u}\cdot \vct{g} \label{EqEng}\,,
 \pard{}{t}\lp \frac{\rho u^2}{2}+\rho\varepsilon +\frac{B^2}{2\mu_0}\rp &=& -\nabla\cdot \lb \rho \vct{u}\lp \frac{u^2}{2}+\varepsilon \rp+\vct{u}\cdot\vct{P}
   -\frac{1}{\mu_0}\vct B\times \vct{E}\rb+\rho \vct{g}\cdot\vct{u}\,,
\end{eqnarray}% 
where $\rho$ is the mass density, $\vct{u}$ is the macroscopic plasma velocity, $\vct E$ and $\vct B$ are electric and magnetic field respectively. We neglected
the electric field term $e n\vct E$ in the momentum equation. The magnitude of electric field is smaller than the magnetic field by the factor $\sim u^2/c^2$.
This system equations needs to be supplied by the state equation for plasma pressure $p=p(\rho,T)$ (where $T$ is a plasma temperature), the Faraday's law
\begin{equation}
 \pard{\vct{B}}{t}=-\nabla\times \vct{E}\,,
\end{equation}
and the Ampere's law in a non--relativistic approximation
\begin{equation}
 \nabla\times\vct{B}=\mu_0\vct{j}\,,
\end{equation}
which binds the current density with the magnetic field and can be considered as a definition of the current density $\vct{j}$ from the computational point of
view. The constant $\mu_0=4\cdot10^{-7}\pi$ is the magnetic permeability. The electric field $\vct E$ is given by Ohm's law
\begin{equation}
 \vct{E}=-\vct{u}\times \vct{B}+\eta \vct{j}\,.
\end{equation}
In the computational physics equations is suitable to have in the conservative form (in the following equation set, the current density is denoted by capital 
$\vct J$ because minuscule $j$ is used as a index).

\begin{eqnarray}
\label{eq:mhd}
 \pard{\rho}{t}+\pard{\rho u_j}{x_j} &=& 0 \,,\nonumber\\
 \pard{(\rho u_i)}{t}+\pard{}{x_j}\lb \rho u_iu_j-\frac{B_iB_j}{\mu_0}+\delta_{ij}\lp p+\frac{\vct{B}^2}{2\mu_0}\rp\rb &=& \rho g_i \,,\nonumber\\
 \pard{B_i}{t}+\pard{}{x_j}\lp u_jB_i-u_iB_j\rp &=& -\pard{}{x_j}\lp\varepsilon_{ijk}\eta J_k\rp \,,\\
 \pard{U}{t}+\pard{}{x_j}S_j &=& \rho g_ju_j \,.\nonumber
\end{eqnarray}

The energy flux $\vct{S}$, and the plasma pressure $p$ for the ideal gas are given by the following relations:
\begin{eqnarray}
U=\frac{p}{\gamma-1}+\frac{1}{2}\rho u^2+\frac{B^2}{2\mu_0}\\
\vct S=\left(U+p+\frac{B^2}{2\mu_0}\right)\vct u- \frac{(\vct
u\cdot\vct B)}{\mu_0}\vct B+\frac{\eta}{\mu_0}\vct j\times\vct B \;.
\end{eqnarray}
In the (almost) collision-less plasma, we are mostly interested in, the classical resistivity plays usually a small role. Instead of that, various microscopical
(kinetic) effects influence the plasma dynamics via other terms in the generalized Ohms law (see \cite{Buchner+Elkina:2006}). In order to mimic these
processes, whose modeling is beyond the scope of the MHD approach, we re-consider the parameter $\eta$ as a generalized resistivity, including the effects like
wave-particle interactions or off-diagonal components in the electron pressure tensor into it. As such effects are -- in general -- observed in the highly
filamented, intense current sheets we model the anomalous generalized resistivity as follows
\begin{equation}
\eta(\vct r,t)=\left\{
  \begin{array}{lll}
    0 & : & |v_{\rm D}|\le v_\mathrm{cr}\\
    C\frac{\left(|v_{\rm D}(\vct r,t)|-v_\mathrm{cr}\right)}{v_0}& : &
    |v_{\rm D}|> v_\mathrm{cr}
  \end{array}
  \right.\,.
  \label{eq:eta}
\end{equation}
Thus, the non-ideal effects are turned on whenever the current-carrier drift velocity
\begin{equation}
 v_{\rm D}(\vct r,t)=\frac{|\vct{j}(\vct r,t)|}{e n_\mathrm{e}}\,,
\end{equation}
exceeds the critical threshold $v_\mathrm{cr}$, which corresponds to the thickness of current sheet.

In order to solve the Eqs.~(\ref{eq:mhd}) numerically, it is convenient to rescale all the quantities to the dimensionless units. Thus, all the spatial
coordinates are expressed in the characteristic size $L_0$ and times in Alfv\'en transit time $\tau_{\rm A}=L_0/v_{\rm A}$, where $v_{\rm A}=B_0/\sqrt{\rho_0}$
is a typical Alfv\'en speed. Magnetic field strength $\vct B$ and plasma density $\rho$ are given in the units of their characteristic values $B_0$ and $\rho_0$
and similar scaling holds for the other quantities -- see \cite{Kliem+:2000} or \cite{Barta+:2011a} for details. The normalization of MHD quantities is
summarized in the table \ref{Tab:MHDNormalization}.

\begin{table}[ht]
  \centering
  \begin{tabular}{|l||c|l|}
    \hline
    Variable  & Normalization & Description \\
    \hline
    \hline
    $L$       & $L_0$ & length \\
    \hline
    $\rho$    & $\rho_0$ & plasma density \\
    \hline
    $\vct{B}$ & $B_0$ & magnetic field \\
    \hline
    $\vct{u}$ & $v_\mathrm{A}=\frac{B_0}{\sqrt{\mu_0\rho_0}}$ & plasma velocity \\
    \hline
    $t$       & $\tau_\mathrm{A}=\frac{L_0}{v_\mathrm{A}}$ & time \\
    \hline
    $p$       & $\frac{B_0^2}{2\mu_0}$ & plasma pressure \\
    \hline
    $\vct{E}$ & $v_\mathrm{A}B_0$ & electric field \\
    \hline
    $\vct{j}$ & $\frac{B_0}{\mu_0L_0}$ & current density \\
    \hline
    $\eta$    & $\mu_0L_0v_\mathrm{A}$ & electric resistivity \\
    \hline
    $U$       & $\frac{1}{2}\rho_0v_\mathrm{A}^2$ & energy \\
    \hline
    $T$       & $\frac{p_0}{k_\mathrm{B}n_0}$ & temperature \\
    \hline
  \end{tabular}
  \caption{Normalization of MHD equations.}
  \label{Tab:MHDNormalization}
\end{table}

In order to utilize more universal LSFEM implementation for the more general form of equations (see \cite{Lukin:2008}) the set of MHD
Eqs.~(\ref{eq:mhd}) is rewritten into the conservative (flux/source) formulation (for the expanded form see Appendix~\ref{sec:exp3DMHDeq})
\begin{equation}
\label{eq:MHDcons}
\pard{\vct{\Psi}}{t}
+\pard{\vct{F}_i(\vct{\Psi},\pard{\vct{\Psi}}{x_j})}{x_i}
=\vct{S}(x_j,\vct{\Psi},t)\,.
\end{equation}
Here the local state vector $\vct{\Psi} = (\rho,\vct{\pi},\vct{B},U)$, the momentum density $\vct{\pi}=\rho\vct{u}$, the flux $\vct{F}$ and the source-term
$\vct{S}$ are defined as
\begin{eqnarray}
\label{eq:MHDFluxSource}
  \vct{F}=\lp
  \begin{array}{c}
    \vct{\pi}\\
    \rho\vct{u}\vct{u}-\vct{B}\vct{B}
    +\vct{\hat{I}_{3\times3}}(p+B^2)\\
    \vct{\hat{\epsilon}_{3\times 3} \cdot E}\\
    (h+E_k)\vct{u}+2\vct{E\times B}\
  \end{array}
  \rp \ ,\qquad
  \vct{S}=\lp
  \begin{array}{c}
     0\\
     \rho\vct{g}\\
     \vct{0}\\
     \vct{\pi \cdot g}
  \end{array}
  \rp \ ,
\end{eqnarray}
where $\vct{\hat{I}_{3\times3}}$ is the $3\times 3$ unit matrix, $\vct{\hat{\epsilon}_{3\times 3}}$ is the permutation pseudo-tensor, $\vct{E}=-\vct{u}\times
\vct{B}+\eta\vct{j}$ is the electric field strength, and the enthalpy and kinetic energy densities are $h=\gamma p/(\gamma-1)$ and $E_k=\rho v^2$, 
respectively.