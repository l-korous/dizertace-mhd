\chapter*{Anotace a klíčová slova}
Předložená téze dizertační práce je zaměřena na pokročilé metody návrhu a analýzy lineárních elektromagnetických aktuátorů s~využitím numerického řešení dílčích matematických modelů pomocí metody konečných prvků. Diskutované metody však nejsou limitovány pouze použitím v~dané problematice, ale je možné je s~výhodou využít také v~široké oblasti modelování a simulací technických problémů, a to nejen z~oblasti elektromagnetizmu.

Tématika elektromagnetických aktuátorů je jednou z~hlavních oblastí zájmů Katedry teoretické elektrotechniky a katedra má v~této problematice nemalé zkušenosti. V~návaznosti na tyto zkušenosti je hlavním cílem práce vytvořit obecnou metodiku použitelnou při návrhu a analýze systémů, které elektromagnetické aktuátory využívají a vytvořit pokročilé nástroje, které umožní řešení dílčích problémů spojených s~danou tématikou.

Práce se zaměřuje na použití parametrických studií, citlivostních analýz a také optimalizací. Aby však bylo možné tyto metody plně využít, je nutné věnovat pozornost numerickým metodám řešení použitých matematických modelů, a to především s~ohledem na jejich stabilitu, přesnost a v~neposlední řadě také časovou náročnost, která je v~mnoha případech značně limitující. 

V~práci jsou využity výhradně vlastní výpočetní prostředky, které jsou výsledkem výzkumné činnosti pracovníků týmu, jehož je autor práce dlouholetým členem. Hlavním využitým nástrojem je pak aplikace Agros2D.

\section*{Klíčová slova}
Elektromagnetické aktuátory, numerická analýza, metoda konečných prvků, parametrická studie, citlivostní analýza, optimalizace, genetické algoritmy, Agros2D.

\chapter*{Anotation and key words}
The thesis is aimed at the advanced methods of design and analysis of linear electromagnetic actuators with utilization of numerical solution of partial mathematical models based on the finite element method. The discussed methods, however, are not limited only by using in this domain, but also in much wider area of modeling and simulations of technical problems beyond electromagnetics.

The topic of electromagnetic actuators is one of the principal interests of the Department of Electrical Engineering and the Department acquired in this domain a very wide experience. Starting from this experience, the main goal of the work is to elaborate a general methodology usable for the design and analysis of systems making use of the electromagnetic actuators and to establish advanced tools that allow solving partial problems connenced with the topic. 

The thesis is aimed at using of parametric studies, sensitivity analysis and also opimization. In order that all the methods can reasonably be used, attention has to be paid to the numerical methods of solution of the mentioned models, mainly with respect to their stability, accuracy, and, last but not least, the computational demands, which represents in a number of cases the limiting factor.

In the thesis I~only use own computational means, which are the results of the research activities of the team of people, where the author is a long-term member. The principal computational tool is the application Agros2D.

\section*{Key words}
Electromagnetic actuators, numerical analysis, finite element method, parametric studies, sensitivity analysis, optimization, genetic algorithms, Agros2D.

\chapter*{Poděkování}
Za skvělé vedení při studiu a práci, časté konzultace a odbornou pomoc zaslouží obrovské díky především prof. Ing. Ivo Doležel, CSc. a doc. Ing. Pavel Karban, Ph.D., kteří se stali mými dlouholetými kolegy a přáteli. Dále bych rád poděkoval všem kolegům se kterými jsem měl tu čest spolupracovat. V~neposlední řadě bych také rád poděkoval své rodině za její neuvěřitelnou výdrž a podporu.

\chapter*{Prohlášení}
Předkládám tímto k~posouzení tezi dizertační práce, zpracovanou na během doktorského studia na Fakultě elektrotechnické Západočeské univerzity v~Plzni.

Prohlašuji, že jsem tuto práci vypracoval samostatně, s~použitím uvedené odborné literatury, a pramenů a že veškerý software, použitý při jejím řešení a zpracování, byl využit s~respektováním všech jeho licenčních podmínek.

\vspace{3em}V Plzni dne 25.4.2014