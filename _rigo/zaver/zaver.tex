\chapter{Zhodnocení a vytyčení dalších směru práce}
V~rámci předložené práce byly diskutovány specifika návrhu lineárních elektromagnetických aktuátorů a byl sestaven obecný matematický model, který je možné využít k~základnímu návrhu nebo analýze konkrétních zařízení. Značná pozornost byla dále věnována specifickým oblastem numerického řešení formulovaného modelu pomocí metody konečných prvků. Mezi tyto oblasti patří
\begin{enumerate}[a]
    \item řešení nelineárních magnetických polí a
    \item výpočty silového působení magnetického pole na feromagnetická tělesa.
\end{enumerate}

Výhody a úskalí diskutovaných metod řešení byly demonstrovány na konkrétním příkladu lineárního elektromagnetického aktuátoru. Ten byl navržen na základě již dříve provedené parametrické studie konstrukce aktuátoru s~ohledem na snahu dosáhnout velkého pracovního rozsahu a také možnost ovlivnit tvar statické charakteristiky provedením tvaru jádra. Aktuátor byl na základě návrhu zkonstruován a experimentálně byla ověřena jeho funkčnost. Výsledky měření statické charakteristiky a teploty byly následně využity pro verifikaci použitého matematického modelu a numerických metod jeho řešení.

Práce byla dále věnována demonstraci moderním přístupům k~návrhu, které umožňují velmi komplexní přístup k~dané problematice. Diskutovanými metodami jsou citlivostní analýza a tvarová optimalizace.

\section{Směr dalších prací}
Další práce v blízké budoucnosti lze rozdělit do dvou směru. První směr bude zaměřen na implementaci modelu napájecího elektrického obvodu do magnetického modulu aplikace Agros2D, jeho testování a následnou analýzu. Dále bude pokračovat vývoj automatického návrhového systému pro aplikaci Agros2D, který umožní snadné vytvoření geometrie navrhovaného lineárního elektromagnetického aktuátoru dle jeho typu a základních parametrů, definici problému, řešení a následnou analýzu návrhu.

Cílem druhého směru prací bude konstrukce nového typu bistabilního elektromagnetického ventilu. Základní koncepce ventilu je v~porovnání s~prakticky používanými zařízeními značně inovativní a přináší mnohé výhody. Základní uspořádání nového zařízení je zobrazeno na obr.\Cref{obr:ventil}.

\begin{figure}[h!]
  \centering
  \includegraphics{zaver/koncept.pdf}
  \caption{Základní koncept navrhovaného elektromagnetického ventilu}
  \label{obr:ventil}
\end{figure}

Hlavní myšlenkou koncepce je využití dutého jádra, kterým řízená kapalina protéká. Navržená koncepce využívá dvou oddělených magnetických obvodů. Magnetický tok vyvolaný budící cívkou v~primárním magnetickém obvodu slouží k~posunutí pohyblivého jádra do otevřeného stavu a uvolnění toku kapaliny. Magnetický indukční tok v~sekundárním obvodu je vytvořen silným permanentním magnetem a slouží k~uzavření ventilu v~případě vypnutí budící cívky.

Mezi hlavní výhody uvedené koncepce patří:
\begin{itemize}
    \item bistabilní režim aktuátoru zajišťující bezpečné uzavření ventilu v~případě ztráty napájecího napětí,
    \item absence konstrukčních prvků pro převod sílového působení, které komplikují čištění ventilu,
    \item využití oddělených magnetických obvodů, která zajišťuje dvě cesty magnetického indukčního toku a je tak zamezeno nežádoucí demagnetizaci permanentního magnetu.
\end{itemize}

\section{Harmonogram dalších prací}
\begin{itemize}
    \item červen 2014: dokončení a zhodnocení koncepční studie pro návrh lineárních elektromagnetických aktuátorů
    \item srpen 2014: dokončení implementace modelu elektrických obvodů do magnetického modulu aplikace Agros2D
    \item září 2014: uzavření návrhu a analýzy elektromagnetického ventilu, dokončení vývoje automatického návrhového systému
    \item říjen 2014: experimentální ověření a diagnostika navrženého ventilu
    \item prosinec 2014: dokončení a odevzdání dizertační práce
    \item březen 2015: obhajoba dizertační práce
\end{itemize}