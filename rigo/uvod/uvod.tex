\chapter{Úvod do problematiky}
Následující kapitola si klade za cíl vymezit oblast zájmu této práce, definovat důležité pojmy, zhodnotit současný stav poznání a také stanovit cíle práce.

\section{Vymezení základních pojmů}
Aktuátor je akční člen, který umožňuje přeměnu vstupní energie na energii mechanickou na jeho výstupu. Na základě způsobu jeho řízení lze rozlišovat dva základní typy, a to konvertor a kontroler. Výstupní mechanickou energii konvertoru (obr. \ref{obr:konvertor}) je možné řídit energií vstupní, kontroler (obr. \ref{obr:kontroler}) umožňuje řízení na základě jiné řídicí veličiny. Výstupní energii aktuátoru je vždy energie mechanická, a to v~podobě silového působení vyvolávajícího pohyb, tlak nebo deformaci. \cite{janocha2010actuators} \cite{ulrych2009aktuatory} \cite{mach2010aktuatory}

\begin{figure}[h!]
  \centering
  \includegraphics{uvod/konvertor.pdf}
  \caption{Blokové schéma aktuátoru typu konvertor}
  \label{obr:konvertor}
\end{figure}

\begin{figure}[h!]
  \centering
  \includegraphics{uvod/kontroler.pdf}
  \caption{Blokové schéma aktuátoru typu kontroler}
  \label{obr:kontroler}
\end{figure}

Aktuátory lze dělit na základě vstupní veličiny, kterou může být tlak kapaliny či plynu, elektrický proud nebo napětí, ale také teplota a jiné fyzikální veličiny. Další dělení je možné podle stupňů volnosti pohyblivých prvků aktuátoru a také podle počtu stabilních poloh. Základní dělení je patrné z~tabulky \ref{tab:deleni_aktuatoru}. \cite{brauer2006magnetic}

\begin{table}[h!]
\caption{Základní dělení aktuátorů}
\begin{center}
\begin{tabular}{|l|c|}
  \hline
  \multirow{4}{*}{Vstupní energie} & hydraulické \\
  & pneumatické \\
  & elektromechanické \\
  & speciální \\
  \hline
  \multirow{4}{*}{Stupně volnosti} & lineární \\
  & rotační \\
  & planární \\
  & kulové \\
  \hline
  \multirow{2}{*}{Počet stabilních poloh} & bistabilní \\
  & monostabilní \\
  \hline
\end{tabular}
\end{center}
\label{tab:deleni_aktuatoru}
\end{table}

Elektromechanické aktuátory využívají jako vstupní veličinu elektrické napětí, nebo proud. Základem jejich funkce je tak vznik silového působení v~elektrickém nebo magnetickém poli. Lze tedy elektromechanické aktuátory dělit na dva základní druhy, a to aktuátory využívající
\begin{enumerate}[a]
    \item silové působení elektrického pole na elektricky nabitá tělesa a dielektrická tělesa,
    \item silové působení magnetického pole na proudovodiče a feromagnetická tělesa.
\end{enumerate}

V~makroskopickém měřítku je energie elektrického pole mnohonásobně menší než energie akumulovaná polem magnetickým. V~mikroskopickém měřítku je však energie elektrického a magnetického pole srovnatelná. Použití principů silového působení elektrického pole se tak prakticky vztahuje jen na mikroaktuátory. \cite{husak2008mikrosenzory}

Vzhledem k~velmi široké definici lze mezi elektromechanické aktuátory řadit všechny elektrické motory a další akční členy. Vzhledem k~tak značné šíři této tématiky bude tato práce dále úzce zaměřená na lineární elektromagnetické aktuátory.

Elektromagnetické aktuátory jsou podmnožinou aktuátorů elektromechanických. Jejich vstupní veličinou je elektrický proud vytvářející v~aktuátoru magnetické pole, které následně vyvolává silové působení na jeho pohyblivé části \cite{boldea2005linear}. Vzhledem ke značnému vývoji v~oblasti materiálů s~obsahem vzácných zemin může být vstupní veličinou vytvářející magnetické pole také remanentní indukce silných permanentních magnetů. Elektromagnetický aktuátor tedy může tedy pracovat jako konvertor, ale také jako kontroler. Obecné blokové schéma elektromagnetického aktuátoru je patrné z~obrázku \ref{obr:obecne_schema_elmag_aktuatoru}. \cite{mach2010aktuatory}

\begin{figure}[h!]
  \centering
  \includegraphics{uvod/obecne_schema_elmag_aktuatoru.pdf}
  \caption{Blokové schéma obecného elektromagnetického aktuátoru ($i$ značí vstupní proud, $B_\mathrm{r}$ remanentní indukci permanentních magnetů a $F$ příslušnou sílu)}
  \label{obr:obecne_schema_elmag_aktuatoru}
\end{figure}

\section{Využití aktuátorů v~soudobé technice}
Lineární elektromagnetické aktuátory jsou nedílnou součástí mnoha komplexních mechatronických systémů, kde jsou používány ve funkci akčních členů. Mezi jejich hlavní výhody patří vysoká dynamika a rychlost reakce, široký rozsah dosahovaného silového působení, jednoduchost, robustnost, snadné řízení a v~neposlední řadě také nízké riziko nepříznivého dopadu na životní prostředí. \cite{gomis2010design}

V~mechatronických soustavách se dnes velmi často můžeme setkat s~elektromagnetickými aktuátory ve funkci výkonových akčních prvků. Výrobci dodávají obzvláště na trhu s~manipulátory a spínači širokou řadu výrobků pro všestranné použití, např. \cite{standardtechnologyinc.com} \cite{etogroup.com}. S~elektromagnetickými aktuátory se lze také setkat u~adaptronických soustav, kde jsou využívány jako akční členy s~vysokou přesností posuvu a nízkými energetickými nároky \cite{4134969} \cite{1252843} \cite{GomisBellmunt2007153} \cite{Lee200024}.

Velmi perspektivní oblastí využití elektromagnetických aktuátorů jsou také ventily, které jsou využívány v~mnoha aplikacích. Například dávkovací ventily, které vyžadují precizní řízení polohy jádra a jeho rychlosti, bezpečnostní ventily pracující v~monostabilním režimu nebo jednoduché ventily s~řízením průtoku kapaliny \cite{nesbitt2011handbook} \cite{5953509}. Moderním využitím jsou také ventily pro vstřikování paliva v~motorech automobilů \cite{1406108} \cite{6531035}.

V~oblasti automobilového průmyslu se kromě využití elektromagnetických ventilů můžeme setkat také s~aktivním závěsným systémem kol \cite{4677555}, nebo například s~elektromagnetickou spojkou, nebo tlumičem \cite{gysen2010active}.

%\cite{tonoli2010design}

\section{Základní typy lineárních\\ elektromagnetických aktuátorů}
Elektromagnetické aktuátory lze rozdělit na aktuátory s~pasivním\footnote{Solenoid actuators} nebo aktivním pohyblivým jádrem\footnote{Moving coil actuators, voice coil actuators}. Pasivní jádro je tvořeno feromagnetickým materiálem, aktivní jádro pak zdrojem magnetického pole, kterým mohou být cívka nebo permanentní magnet. Samozřejmě lze oba uvedené typy libovolně kombinovat. Základní běžně používaná uspořádání jsou patrná z~obr. \ref{obr:zakladni_typy}. \cite{gomis2010design} \cite{ulrych2009aktuatory}

\begin{figure}[h!]
  \centering
  \includegraphics{uvod/zakladni_typy.pdf}
  \caption{Základní typy lineárních elektromagnetických aktuátorů}
  \label{obr:zakladni_typy}
\end{figure}

\section{Současný stav poznání}
Při magnetickém návrhu elektromagnetických aktuátoru je v~současné době velmi často využívána metoda řešení magnetických obvodů na základě jejich ekvivalence s~elektrickými obvody\footnote{Magnetic equivalent circuit, reluctance method}. Metoda umožňuje uvažovat jak lineární, tak nelineární prostředí. Velkou výhodou metody je snadná formulace matematického modelu a zároveň jeho řešení. Díky tomu je také možné provádět optimalizaci návrhu. \cite{gomis2010design} \cite{4270652}

Metoda má však také značná omezení, mezi která patří především nutnost provést často velké zjednodušení řešeného návrhu tak, aby bylo možné ekvivalenci provést. V~takovém případě to má za následek omezení složitosti návrhu, a to především v~geometrii daného aktuátoru. Další nevýhodou může být také velmi komplikované začlenění rozptylových magnetických toků, nebo provázání modelu s~modely respektujícími další fyzikální děje (přenos tepla, proudění tekutin, atd.). \cite{gomis2010design}

Používané metody mají tedy značný dopad na složitost řešených návrhů. Lineární elektromagnetické aktuátory jsou tak často konstruovány jako velmi jednoduchá zařízení, která disponují jen velmi malým pracovním rozsahem. Síla, která působí na pohyblivé jádro takového aktuátoru je silně závislá na jeho poloze a tím je ovlivněna také dynamika jeho pohybu. Velkou nevýhodou u~používaných aktuátorů jsou také jejich rozměry.

Rychlý vývoj numerických metod a nástrojů pro modelování fyzikálních polí se projevuje samozřejmě také v~diskutované oblasti elektromagnetických aktuátorů. Nasazení této metodiky však často naráží na složitost používaných matematických modelů či numerických metod a také na značné časové nároky dílčích výpočtů. \cite{brauer2006magnetic}

Výhody, které však uvedené metody přinášejí, otevírají široké možnosti přístupu k~dané problematice a umožňují vytvářet velmi komplexní návrhy a zároveň také provádět simulace jejich funkčnosti v~mnoha provozních stavech. \cite{5484675} \cite{6265943}

\section{Motivace a cíle práce}
Z~pohledu modelování představuje matematický model elektromagnetického aktuátoru velmi komplexní úlohu, jejíž korektní řešení není nikterak triviální. Model musí popisovat všechny důležité fyzikální procesy, které v~aktuátoru nastávají. Obecný model tedy zahrnuje
\begin{itemize}
    \item přechodné děje ve výkonových nelineárních elektrických obvodech,
    \item vznik a časový vývoj magnetického pole v~lineárním i nelineárním prostředí aktuátoru,
    \item velmi rychlé nelineární děje popisující dynamiku pohybu pohyblivých částí aktuátoru a
    \item setrvačný ohřev a přestup tepla uvnitř aktuátoru.
\end{itemize}

Při formulaci modelu je také často nutné uvažovat další specifické fyzikální procesy. Jedná se například o~výpočty zatěžování vlivem proudění kapalin a plynů, termoelastické deformace dílčích částí aktuátoru, nebo například vliv změn vstupních veličin v~důsledku řízení aktuátoru a celého mechatronického systému.

Obecný elektromagnetický aktuátor tedy popisuje sdružený nelineární model fyzikálních polí, elektrických obvodů a dynamických dějů, který je nutné řešit jako transientní úlohu na různých časových škálách (multiscale modeling). Zároveň je často nutné uvažovat značnou nepřesnost ve vstupních parametrech, kterými jsou například materiálové vlastnosti nebo zatěžovací charakteristiky a při řešení modelu je tedy v~mnoha případech velmi důležité věnovat pozornost jeho citlivostní analýze.

Vzhledem k~velkému množství řešení, které se při návrhu elektromagnetických aktuátorů nabízejí, je zároveň velmi těžké nalézt řešení optimální s~ohledem na stanovené vlastnosti daného aktuátoru a jeho aplikaci. Kromě požadované velikosti silového působení a pracovního rozsahu aktuátoru se často jedná také o~požadavek na konstantní silové účinky v~celém pracovním rozsahu, minimální rozměry aktuátoru a další specifické požadavky.

Při hledání optimálního řešení však mohou značně pomoci moderní optimalizační algoritmy, které umožňují využít více cílových funkcionálů, tedy zahrnout do optimalizačního procesu více kritérií, které z~hlediska předem stanovených vlastností aktuátoru hodnotí dané varianty (multi-kriteriální optimalizace).

Hlavním cílem této práce je tedy vytvoření obecně použitelné metodiky pro návrh elektromagnetických aktuátorů s~využitím pokročilých metod návrhu založených na numerickém řešení komplexního matematického modelu. Mezi dílčí cíle práce pak patří
\begin{itemize}
    \item systematická a kritická studie elektromagnetických aktuátorů,
    \item vývoj dostatečně komplexního nástroje umožňujícího numerické řešení obecného matematického modelu,
    \item vývoj pokročilých metod využitelných při návrhu elektromagnetických aktuátorů,
    \item demonstrace využití metodiky na konkrétních příkladech.
\end{itemize}