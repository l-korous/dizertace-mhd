\section{Boundary conditions}
\label{section:bcs}
For the problem of finding the solution \ref{weakSlnDef} to be complete, we need to specify the proper boundary conditions.
\subsection{Essential (inflow) boundary conditions}

Since the solution \ref{weakSlnDef} of the problem specified in \ref{WeakFinal} is not influenced by its values on the boundary $\Omega_t$, we are not able to employ standard essential boundary conditions of the form $\bfu\left(x, y, z\right) = \bfu_D\left(x, y, z\right)$ with a known $\bfu_D$.
If such a condition is required from the physical nature of the described phenomenon (as often is the case), it is only implied by the use of fluxes - as one can see in the last integrand $\mrF\lo\mrPsi\ro \mrv$ in \ref{WeakFinal}.


\subsection{Outflow (do-nothing) boundary conditions}
If we want to model free boundary, that is not in any way present as an actual physical boundary or interface, this is usually achieved by specifying the fluxes through the boundary $\mrF\lo\mrPsi\ro \mrv$ in \ref{WeakFinal} to be zero:
$$
\mrF\lo\mrPsi\lo\bfx\ro\ro = 0\ \forall\bfx\in\partial\Omega_t\, \forall t\in \lo0, T\ro.
$$


\subsection{Periodic boundary conditions}
Periodic boundary condition is always specified on two parts $\Gamma_1, \Gamma_2$ of the domain boundary that share the bijection mapping between points:
\be
\label{periodicMapping}
\left[x_1, y_1, z_1\right] \leftrightarrow \left[x_2, y_2, z_2\right] \forall \left[x_1, y_1, z_1\right] \in \Gamma_1,\ \forall \left[x_2, y_2, z_2\right] \in \Gamma_2,
\ee
so that for each pair of related points $\left[x_1, y_1, z_1\right] \leftrightarrow \left[x_2, y_2, z_2\right]$, the values must be the same:
\be
\label{periodicBCs}
u\left(\left[x_1, y_1, z_1\right]\right) = u\left(\left[x_2, y_2, z_2\right]\right) \forall \left[x_1, y_1, z_1\right] \in \Gamma_1,\ \forall \left[x_2, y_2, z_2\right] \in \Gamma_2:\, \left[x_1, y_1, z_1\right] \leftrightarrow \left[x_2, y_2, z_2\right].
\ee
