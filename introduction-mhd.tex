\section{Magnetohydrodynamics}
The term Magnetohydrodynamics covers all physical phenomena that involve both electromagnetic (EM) field and a fluid that carries the EM field are very interesting, yet very complex to study. The behavior of such a fluid is utilized in some industrial applications - liquid-metal cooling of nuclear reactors, magnetic fluid in dampers, sensors for precise measuring of angular velocities, etc. Such phenomena occur in nature as well - the most significant of which are definitely the processes that take place in and on stars - which is the topic of next subsection.