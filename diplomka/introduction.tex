\addcontentsline{toc}{chapter}{Introduction}

\begin{flushleft}
{\LARGE{\textbf{{Introduction}}}}
\end{flushleft}

Discontinuous Galerkin methods (DG methods) in mathematics form a class of numerical methods for solving partial differential equations. They combine features of the finite element and the finite volume framework and have been successfully applied to hyperbolic, elliptic and parabolic problems arising from a wide range of applications. DG methods have in particular received considerable interest for problems with a dominant first-order part, e.g. in electrodynamics, fluid mechanics and plasma physics.

Discontinuous Galerkin methods were first proposed and analyzed in the early 1970s as a technique to numerically solve partial differential equations. In 1973 Reed and Hill introduced a DG method to solve the hyperbolic neutron transport equation.

The origin of the DG method for elliptic problems cannot be traced back to a single publication as features such as jump penalization in the modern sense were developed gradually. However, among the early influential contributors were Babu�ka, J.-L. Lions, Nitsche and Zlamal. Interestingly DG methods for elliptic problems were already developed in a paper by Baker in the setting of 4th order equations in 1977. A more complete account of the historical development and an introduction to DG methods for elliptic problems is given in a publication by Arnold, Brezzi, Cockburn and Marini. A number of research directions and challenges on DG methods are collected in the proceedings volume edited by Cockburn, Karniadakis and Shu.

The discontinuous Galerkin (DG) methods are locally conservative, stable, and high-order accurate methods which can easily
handle complex geometries, irregular meshes with hanging nodes, and approximations that have polynomials of different
degrees in different elements. These properties, which render them ideal to be used with hp-adaptive strategies, not only
have brought these methods into the main stream of computational fluid dynamics, for example, in gas dynamics, compressible and incompressible flows, turbomachinery, magneto-hydrodynamics,
granular flows, semiconductor device simulation, etc., but have also prompted their application
to a wide variety of problems for which they were not originally intended like, for example, second-order elliptic problems, and elasticity.

\ \\Some more stuff here, references, etc...