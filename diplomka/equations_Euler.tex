\section{Euler equations and their properties}
\label{sec:euler_properties}
If we set $\mu = \lambda = k = 0$, we obtain the model of inviscid compressible flow,
described by the continuity equation, the Euler equations, the energy equation and
thermodynamical relations. Since gases are light, usually it is possible to neglect the
effect of the volume force. Neglecting heat sources too (assuming $adiabatic flow$), we get the system for the perfect inviscid gas in the following form:
\begin{eqnarray}
  \pds{\rho}{t} + \mathrm{div}\left(\rho\bs{v}\right) & = & 0,\\
  \pds{\left(\rho\bs{v}\right)}{t} + 
  \mathrm{div}\left(\rho\bs{v}\otimes\bs{v}\right)+ 
  \nabla p & = & 0,\\
  \pds{E}{t} + \mathrm{div}\lo\lo E+p\ro\bs{v}\ro & = & 0,\\
   p & = & \lo\gamma-1\ro\lo{E}-\rho\left|\bs{v}\right|^2/2\ro.
	\end{eqnarray}
This system is simply called the $compressible\ Euler\ equations$.
\paragraph{}
In the following we shall be concerned with the 2-dimensional case, which will be used in numerical examples. All results apply for the 3-dimensional case as well.

\subsection{Dimensionless Euler equations}
We set 3 similarity constants $l_r, v_r,$ and $\rho_r$, which are, in order: characteristic length, characteristic velocity, and characteristic density. Now we multiply the Euler equations by these constants in the following way:

\begin{eqnarray}
\left[{\partial\rho\over\partial t} + \mathrm{div}(\rho{\bs{v}})\right] {l_r\over\rho_r v_r} & = & 0,\\ \left[{\partial(\rho{\bs{v}})\over\partial t} + \mathrm{div}(\rho{\bs{v}}\otimes{\bs{v}}) + \nabla p\right]
{l_r\over\rho_r v_r^2} & = & 0,\\
\left[{\partial E\over\partial t} +\mathrm{div}((E+p){\bs{v}})\right] {l_r\over\rho_r v_r^3} & = & 0.
\end{eqnarray}

Then we get the dimensionless Euler equations in the form

\begin{eqnarray}
{\partial\tilde\rho\over\partial \tilde t} + \tilde\nabla\cdot(\tilde\rho \tilde{\bs{v}}) & = & 0,  \\ {\partial(\tilde \rho\tilde{\bs{v}})\over\partial \tilde t} + \tilde\nabla\cdot(\tilde\rho\tilde{\bs{v}}\otimes\tilde{\bs{v}}) + \tilde\nabla\tilde p & = &  0, \\ {\partial\tilde E\over\partial\tilde t} + \tilde\nabla\cdot( (\tilde E+\tilde p)\tilde{\bs{v}}) & = &  0,
\end{eqnarray}

where

\begin{eqnarray}
t_r & = & {l_r\over v_r},\\
\tilde t & = &  {t\over t_r},\\
\tilde \rho & = &  {\rho\over\rho_r},\\
\tilde{\bs{v}} & = &  {{\bs{v}}\over v_r},\\
\tilde\nabla & = &  l_r\nabla,\\
\tilde E & = &  {E\over \rho_r u^2_r},\\
\tilde p & = &  {p\over \rho_r u^2_r}.
\end{eqnarray}

As we see, the dimensionless Euler equations have the same form as the original equations, only the relations are rescaled using the above relations. Therefore, in the sequel, we shall omit the symbol "~" over the considered quantities.

\subsection{Conservative form of the Euler equations}
The system of governing equations can be written in the so-called $conservative$ $form$.
\be
\label{conservative_form}
{\partial\bs{w}\over \partial t} + {\partial{\bs{f}}_1\over \partial x_1} + {\partial{\bs{f}}_2\over \partial x_2} = 0,
\ee

where

\begin{eqnarray}
{\bs{w}} & = & \left( \begin{array}{c} \varrho\\ \rho v_1\\ \rho v_2\\ E \end{array} \right) = \left( \begin{array}{c} w_1 \\ w_2 \\ w_3 \\ w_4 \\ \end{array} \right), \\
\label{f_1}
{\bs{f}}_1 & = & \left( \begin{array}{c} \rho v_1\\ \rho v_1^2 + p\\ \rho v_1 v_2\\ v_1(E+p) \end{array} \right) = \left( \begin{array}{c} w_2\\ \frac{w_2^2}{w_1} + p\\ \frac{w_2w_3}{w_1}\\ \frac{w_2}{w_1}(w_4+p) \end{array} \right), \\
\label{f_2}
{\bs{f}}_2 & = & \left( \begin{array}{c} \rho v_2\\ \rho v_2 v_1\\ \rho v_2^2 + p\\ v_2(E+p) \end{array} \right) = \left( \begin{array}{c} w_2\\ \frac{w_3w_2}{w_1}\\ \frac{w_3^2}{w_1} + p\\ \frac{w_3}{w_1}(w_4+p) \end{array} \right), \\
p & = & \lo{\gamma - 1}\ro \lo w_4 - \frac{w_2^2 + w_3^2}{2w_1}\ro.
\end{eqnarray}
Usually, ${\bs{f}}_1$ and ${\bs{f}}_2,$ are called $inviscid\ Euler\ fluxes$. The unknowns $\rho, v_1, v_2, p$ in the original system of equations are called $primitive\ variables$, whereas $w_1 = \rho, w_2 = \rho v_1, w_3 = \rho v_2, w_4 = E$ are called $conservative\ variables$.
\paragraph{}
The domain of definition of the vector-valued functions $\bs{f}_1, \bs{f}_2$ is the open set $D\subset\mathrm{R}^2$ where the corresponding density and pressure are positive:
\be
D=\left\{\bs{w}\in\mathrm{R}^4;\ w_1 = \rho > 0, w_4 - \frac{w_2^2 + w_3^2}{2w_1} = \frac{p}{\gamma - 1} > 0\right\}.
\ee
One can see that the inviscid Euler fluxes are continuously differentiable in their domain of definition. Differentiation in~\eqref{conservative_form} and the chain rule lead to a first order system of quasilinear partial differential equations
\be
\label{final_Euler_form}
\pds{\bs{w}}{t} + \mathbb{A}_1\lo\bs{w}\ro \pds{\bs{w}}{x_1} + \mathbb{A}_2\lo\bs{w}\ro \pds{\bs{w}}{x_2} = 0,
\ee
where $\mathbb{A}_s\lo \bs{w} \ro, s = 1, 2$ are $ m \times m$ matrices called $flux\ Jacobians$ defined as the Jacobi matrices of the mappings $\bs{f}_s, s = 1, 2$, at a point $\bs{w}\in D$:
\be
\mathbb{A}_s\lo\bs{w}\ro = \frac{D\bs{f}_s\lo\bs{w}\ro}{D\bs{w}} = \lo\frac{\partial f_{si}\lo\bs{w}\ro}{\partial w_{j}}\ro_{i, j = 1}^{2}.
\ee
For $\bs{w}\in D$ and $\bs{n} = \lo n_1, n_2\ro^T\in\mathbb{R}^2$ we put
\be
\label{P_definition}
\mc{P}\lo\bs{w},\bs{n}\ro = \sum_{s = 1}^{2}\bs{f}_s\lo\bs{w}\ro n_s,
\ee
which is the flux of the quantity $\bs{w}$ in the direction $\bs{n}$. The Jacobi matrix $D\mc{P}\lo\bs{w},\bs{n}\ro/D\bs{w}$ can be expressed in the form
\be
\frac{D\mc{P}\lo\bs{w},\bs{n}\ro}{D\bs{w}} = \mathbb{P}\lo\bs{w},\bs{n}\ro = \sum_{s=1}^2\mathbb{A}_s\lo\bs{w}\ro n_s.
\ee
\paragraph{Lemma 1.5}
\ \\
\itshape
The vector-valued functions $f_s, s = 1, 2$, defined by~\eqref{f_1} -~\eqref{f_2}, are first order homogenous mappings:
\be
\label{first_homo}
\bs{f}_s\lo\alpha\bs{w}\ro = \alpha\bs{f}_s\lo\bs{w}\ro,\ \alpha > 0.
\ee
Moreover, we have
\be
\bs{f}_s\lo\bs{w}\ro = \mathbb{A}_s\lo\bs{w}\ro\bs{w}.
\ee
Similarly,
\begin{eqnarray}
\mc{P}\lo\alpha\bs{w},\bs{n}\ro & = & \alpha\mc{P}\lo\bs{w},\bs{n}\ro,\ \alpha > 0,\\
\mc{P}\lo\bs{w},\bs{n}\ro & = & \mathbb{P}\lo\bs{w},\bs{n}\ro\bs{w}.
\end{eqnarray}
\upshape
\paragraph{Proof:}
Relation~\eqref{first_homo} immediately follows from~\eqref{f_1} -~\eqref{f_2}. Since $\bs{f}_s\in \mc{C}^1\lo D\ro^m$, the expression $\lo D \bs{f}_s\lo\bs{w}\ro/ D \bs{w}\ro \bs{w} = \mathbb{A}_s\lo\bs{w}\ro\bs{w}$ is the derivative of $\bs{f}_s$ in the direction $\bs{w}$ at the point $\bs{w}$. By the definition of the derivative and~\eqref{first_homo},
\begin{eqnarray}
\mathbb{A}_s\lo\bs{w}\ro\bs{w} & = & \lim_{\alpha\rightarrow 0}\frac{\bs{f}_s\lo\bs{w} + \alpha\bs{w}\ro - \bs{f}_s\lo\bs{w}\ro}{\alpha}\\
& = & \lim_{\alpha\rightarrow 0}\frac{\lo 1 + \alpha\ro \bs{f}_s\lo\bs{w}\ro - \bs{f}_s\lo\bs{w}\ro}{\alpha} = \bs{f}_s\lo\bs{w}\ro.
\end{eqnarray}
The rest of the Lemma follows from the definitions of $\mc{P}$ and $\mathbb{P}$ and the above results.
The flux Jacobians have the following form:
\be
\nonumber
  \mathbb{A}_1({\bs{w}}) = {D{\bs{f}}_1\over D {\bs{w}}}
\ee
\scriptsize
\begin{eqnarray}
\nonumber
= \left( \begin{array}{cccc} 0 & 1 & 0 & 0\\ -{w_2^2\over w_1^2} +{R\over c_v}{w_2^2+w_3^2\over 2 w_1^2} & {2w_2\over w_1}-{R\over c_v}{w_2\over w_1} & -{R\over c_v}{w_3\over w_1} & {R\over c_v}\\ -{w_2w_3\over w_1^2} & {w_3\over w_1} & {w_2\over w_1} & 0 \\ -{w_2w_4\over w_1^2}-{w_2\over w_1^2}{R\over c_v} \left(w_4-{w_2^2+w_3^2\over 2 w_1}\right) +{w_2\over w_1}{R\over c_v}{w_2^2+w_3^2\over 2 w_1^2} & {w_4\over w_1}+{1\over w_1}{R\over c_v} \left(w_4-{w_2^2+w_3^2\over 2 w_1}\right) -{R\over c_v}{w_2^2\over w_1^2} & -{R\over c_v}{w_2w_3\over w_1^2} & {w_2\over w_1}+{R\over c_v}{w_2\over w_1} \\ \end{array} \right),
\end{eqnarray}
\normalsize
\be
\nonumber
\mathbb{A}_2({\bs{w}}) = {D{\bs{f}}_2\over D {\bs{w}}}
\ee
\scriptsize
\begin{eqnarray}
\nonumber
= \left( \begin{array}{cccc} 0 & 0 & 1 & 0\\ -{w_3w_2\over w_1^2} & {w_3\over w_1} & {w_2\over w_1} & 0 \\ -{w_3^2\over w_1^2} +{R\over c_v}{w_2^2+w_3^2\over 2 w_1^2} & -{R\over c_v}{w_2\over w_1} & {2w_3\over w_1} -{R\over c_v}{w_3\over w_1} & {R\over c_v}\\ -{w_3w_4\over w_1^2}-{w_3\over w_1^2}{R\over c_v} \left(w_4-{w_2^2+w_3^2\over 2 w_1}\right) +{w_3\over w_1}{R\over c_v}{w_2^2+w_3^2\over 2 w_1^2}& -{R\over c_v}{w_3w_2\over w_1^2} & {w_4\over w_1}+{1\over w_1}{R\over c_v} \left(w_4-{w_2^2+w_3^2\over 2 w_1}\right) -{R\over c_v}{w_3^2\over w_1^2} & {w_3\over w_1}+{R\over c_v}{w_3\over w_1} \\ \end{array} \right).
\end{eqnarray}
\normalsize

\subsection{Rotational invariance}
\label{sec:rot_inv}
The rotational invariance of the Euler equations is represented by the relations
\begin{eqnarray}
\mc{P}\lo\bs{w},\bs{n}\ro & = & \sum_{s=1}^2\bs{f}_s\lo\bs{w}\ro n_s = \mathbb{Q}^{-1}\lo\bs{n}\ro\bs{f}_1\lo\mathbb{Q}\lo\bs{n}\ro\bs{w}\ro,\\
\label{rot-P}\mathbb{P}\lo\bs{w},\bs{n}\ro & = & \sum_{s=1}^2\mathbb{A}_s\lo\bs{w}\ro n_s = \mathbb{Q}^{-1}\lo\bs{n}\ro\mathbb{A}_1\lo\mathbb{Q}\lo\bs{n}\ro\bs{w}\ro\mathbb{Q}\lo\bs{n}\ro,\\
\bs{n} & = & \lo n_1, n_2\ro \in \mathbb{R}^2, \left|\bs{n}\right| = 1, \bs{w} \in D,
\end{eqnarray}
where
\be
\label{Q}
\mathbb{Q}\lo\bs{n}\ro = \lo\begin{array}{cccc}
1 & 0 & 0 & 0 \\
0 & n_1 & n_2 & 0\\
0 & -n_2 & n_1 & 0\\
0 & 0 & 0 & 1
\end{array}
\ro.
\ee

\subsection{Diagonalization of the Jacobi matrix}
We have
\begin{eqnarray}
\bs{f}_1\lo\bs{w}\ro = \lo w_1, w_2^2/w_1 +\lo \gamma - 1\ro\left[w_4 - \lo w_2^2 + w_3^2\ro/\lo 2w_1\ro\right],\right.\\\nonumber
\left.w_2w_3/w_1,w_2\left[\gamma w_4 - \lo\gamma - 1\ro\lo w_2^2 + w_3^2\ro/\lo 2w_1\ro\right]/w_2\ro^T,
\end{eqnarray}
and, with the notation $\bs{v} = \lo u, v\ro$,
\be
\mathbb{A}_1\lo\bs{w}\ro = \lo\begin{array}{cccc}
0 & 1 & 0 & 0\\
\frac{\gamma - 1}{2}\left|\bs{v}\right|^2 - u^2 & \lo 3 - \gamma\ro u &\lo 1 - \gamma\ro v & 1 - \gamma \\
-uv & v & u & 0\\
u\lo\lo\gamma -1\ro\left|\bs{v}\right|^2 - \gamma\frac{E}{\rho}\ro & \gamma\frac{E}{\rho} - \lo\gamma - 1\ro u^2 - \frac{\gamma - 1}{2}\left|\bs{v}\right|^2 & \lo 1 - \gamma\ro u v & \gamma u
\end{array}\ro.
\ee
The matrix $\mathbb{A}_1\lo\bs{w}\ro$ has eigenvalues
\be
\tilde \lambda_1\lo\bs{w}\ro = u - a, \tilde\lambda_2\lo\bs{w}\ro = \tilde\lambda_3\lo\bs{w}\ro = u, \tilde\lambda_4\lo\bs{w}\ro = u + a
\ee
and the corresponding eigenvectors are
\begin{eqnarray}
\bs{r}_1\lo\bs{w}\ro & = & \lo 1, u - a, v, \left|\bs{v}\right|^2/2 + a^2\lo\gamma - 1\ro - ua\ro^T,\\
\bs{r}_2\lo\bs{w}\ro & = & \lo 1, u, v \left|\bs{v}\right|^2/2\ro^T,\\
\bs{r}_3\lo\bs{w}\ro & = & \lo 1, u, v - a, \left|\bs{v}\right|^2/2-va\ro^T,\\
\bs{r}_4\lo\bs{w}\ro & = & \lo 1, u + a, v, \left|\bs{v}\right|^2/2 + a^2\lo\gamma - 1\ro + ua\ro^T.
\end{eqnarray}
These calculations are possible to be verified. As the eigenvectors $\bs{r}_s\lo\bs{w}\ro, s = 1, ..., 4$ are linearly independent for each $\bs{w}\in D$, we can write
\be
\label{diagonalization}
\tilde{\mathbb{T}}^{-1}\lo\bs{w}\ro\mathbb{A}_1\lo\bs{w}\ro\tilde{\mathbb{T}}\lo\bs{w}\ro = \tilde{\Lambda}\lo\bs{w}\ro,
\ee
where the matrix $\tilde{\mathbb{T}}\lo\bs{w}\ro$ has the vectors $\bs{r}_s\lo\bs{w}\ro, s = 1, ..., 4$ as its columns and
\be
\tilde{\Lambda}\lo\bs{w}\ro = \mathrm{diag}\lo\tilde\lambda_1\lo\bs{w}\ro, ..., \tilde\lambda_4\lo\bs{w}\ro\ro.
\ee
This means that the matrix $\mathbb{A}_1\lo\bs{w}\ro$ is diagonalizable.
\paragraph{}
Now we can show that the matrix $\mathbb{P}\lo\bs{w},\bs{n}\ro$ is diagonalizable too.
By~\eqref{rot-P} and~\eqref{diagonalization}
\be
\label{final_p_diag}
\mathbb{P}\lo\bs{w},\bs{n}\ro = \left|\bs{n}\right|\mathbb{Q}^{-1}\lo\bs{n}\ro\tilde{\mathbb{T}}\lo\mathbb{Q}\lo\bs{n}\ro\bs{w}\ro\tilde{\Lambda}\lo\mathbb{Q}\lo\bs{n}\ro\bs{w}\ro\tilde{\mathbb{T}}\lo\mathbb{Q}\lo\bs{n}\ro\bs{w}\ro^{-1}\mathbb{Q}\lo\bs{n}\ro.
\ee
Under the notation
\begin{eqnarray}
\nonumber
\Lambda\lo\bs{w},\bs{n}\ro = \left|\bs{n}\right|\tilde{\Lambda}\lo\mathbb{Q}\lo\bs{n}\ro\bs{w}\ro = \mathrm{diag}\left|\bs{n}\right|\lo\tilde{\Lambda_1}\lo\mathbb{Q}\lo\bs{n}\ro\bs{w}\ro, ..., \tilde{\Lambda_4}\lo\mathbb{Q}\lo\bs{n}\ro\bs{w}\ro\ro,\\
\mathbb{T}\lo\bs{w},\bs{n}\ro = \mathbb{Q}^{-1}\lo\bs{n}\ro\tilde{\mathbb{T}}\lo\mathbb{Q}\lo\bs{n}\ro\bs{w}\ro,
\end{eqnarray}
we see that the matrix $\mathbb{P}\lo\bs{w},\bs{n}\ro$ satisfies the relation $\mathbb{T}^{-1}\mathbb{P}\mathbb{T} = \Lambda\lo\bs{w},\bs{n}\ro = \mathrm{diag}\lo \lambda_1, ..., \lambda_4\ro$ and thus $\mathbb{T}$ diagonalizes the matrix $\mathbb{P}$.
