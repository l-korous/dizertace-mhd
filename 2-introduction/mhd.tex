The term magnetohydrodynamics (MHD) covers all physical phenomena that involve both electromagnetic (EM) field and a fluid that carries the EM field. Such phenomena are very interesting, yet very complex to study. The behavior of such a fluid is utilized in some industrial applications - liquid-metal cooling of nuclear reactors, magnetic fluid in dampers, sensors for precise measuring of angular velocities, etc. Such phenomena occur in nature as well - the most significant of which are definitely the processes that take place inside and on the surface of stars - which is the topic of the next section.