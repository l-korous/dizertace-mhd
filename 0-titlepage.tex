\begin{titlepage}
    \begin{center}
        
        \vspace*{3em}
        \textsc{\LARGE University of West Bohemia in Pilsen}\\[0.5em]
        \textsc{\LARGE Faculty of Electrical Engineering}\\[5em]
        
        {\Large Doctoral thesis}\vspace*{1em}
        \rule{\linewidth}{0.5mm}
        \\[1em]
        {\huge \textsc{Large-scale Numerical Simulations of Magneto-Hydrodynamics Phenomena in Astrophysics} \\[0.4cm]}
        \rule{\linewidth}{0.5mm}
        \\[3em]
        
        {\Large Mgr. Luk� \textsc{Korous}}
				\end{center}
				\ \\ TODO
				\\ \ \\
				\\ do uvodu napsat vice o state-of-the-art, odkaz na Mirovo clanek, ukazat tam ty FD vysledky, ze je chceme nahradit a ze je tam hodne dulezita adaptivity
				\\ pak teda popsat az tam z toho budu davat vysledky, tak vykopirovat nejaky rovnice IC / BC z clanku, jak se to implementuje, atd.
				\\ srovnani
				\\\ 
				\\ adaptivita - ukazat na 2d prikladu z Hermesu co to je referencni reseni, co to je ||zprojektovane - presne|| (v dealu?), co to je tohle bez normy, zminka o distribuovanosti, ze musim napocitat nejaky thresholdy mapReducem, atd.
				\\ napsat pak algoritmus tedy cely, kde je casovy krok, adaptivita (ze se vraci casovy krok po zjemneni / zhrubeni, aby se neztracela informace), atd.
				\\ \ \\
				\\ popis a vysledky Blastu, O-T
				\\ - bez adaptivity jen "pocita to"
				\\ - s adaptivitou "a navic tak presne to spoctu a usetrim 80\% dofu"
				\\ -- tohle ukazovat na rezech hustotou - bylo by fajn nejaky najit
				\vfill
				\begin{center}
        {\large 2016}
    \end{center}
\end{titlepage}