\subsection{Numerical simulation of MHD Phenomena}
Only recently, the scientific computation community, due to the advances in computer and supercomputer capabilities, has started with non-trivial numerical simulations of such complex physical phenomena that the MHD model describes. Since both for industrial applications, and obviously for astrophysical application of the MHD model, it is quite expensive (or downright impossible) to perform any experiments, the benefit of being able to simulate the phenomena on a computer is very large.\\
There exist several available numerical simulation codes, such as \cite{athena}, \cite{zeus}, \cite{ramses}. These codes have been successfully applied to a range of problems in astrophysics. The reason for the development of a new code is two-fold. First, there is a unique collaboration between the Astronomical Institute of the Czech Academy of Sciences and the University of West Bohemia, where astro-physicists work together with electrical engineers (from theoretical and numerical modeling backgrounds), and the developed code will be usable for both simulating of astrophysical MHD phenomena, and industrial MHD applications. Second, the newly developed code is based on locally-adaptive Discontinuous Galerkin method, which yields several advantages (discussed in the respective section) over the existing codes (using e.g. Finite Difference, or Finite Volumes methods) developed at institutions of such high quality as \emph{Princeton} - \cite{athena}, \cite{zeus}. Another benefit (namely over \cite{zeus})of the newly created software are the use of modern object-oriented programming techniques (the software is written in the C++ language).